% Options for packages loaded elsewhere
\PassOptionsToPackage{unicode}{hyperref}
\PassOptionsToPackage{hyphens}{url}
\PassOptionsToPackage{dvipsnames,svgnames,x11names}{xcolor}
%
\documentclass[
  letterpaper,
]{article}

\usepackage{amsmath,amssymb}
\usepackage{iftex}
\ifPDFTeX
  \usepackage[T1]{fontenc}
  \usepackage[utf8]{inputenc}
  \usepackage{textcomp} % provide euro and other symbols
\else % if luatex or xetex
  \usepackage{unicode-math}
  \defaultfontfeatures{Scale=MatchLowercase}
  \defaultfontfeatures[\rmfamily]{Ligatures=TeX,Scale=1}
\fi
\usepackage{lmodern}
\ifPDFTeX\else  
    % xetex/luatex font selection
\fi
% Use upquote if available, for straight quotes in verbatim environments
\IfFileExists{upquote.sty}{\usepackage{upquote}}{}
\IfFileExists{microtype.sty}{% use microtype if available
  \usepackage[]{microtype}
  \UseMicrotypeSet[protrusion]{basicmath} % disable protrusion for tt fonts
}{}
\makeatletter
\@ifundefined{KOMAClassName}{% if non-KOMA class
  \IfFileExists{parskip.sty}{%
    \usepackage{parskip}
  }{% else
    \setlength{\parindent}{0pt}
    \setlength{\parskip}{6pt plus 2pt minus 1pt}}
}{% if KOMA class
  \KOMAoptions{parskip=half}}
\makeatother
\usepackage{xcolor}
\setlength{\emergencystretch}{3em} % prevent overfull lines
\setcounter{secnumdepth}{5}
% Make \paragraph and \subparagraph free-standing
\makeatletter
\ifx\paragraph\undefined\else
  \let\oldparagraph\paragraph
  \renewcommand{\paragraph}{
    \@ifstar
      \xxxParagraphStar
      \xxxParagraphNoStar
  }
  \newcommand{\xxxParagraphStar}[1]{\oldparagraph*{#1}\mbox{}}
  \newcommand{\xxxParagraphNoStar}[1]{\oldparagraph{#1}\mbox{}}
\fi
\ifx\subparagraph\undefined\else
  \let\oldsubparagraph\subparagraph
  \renewcommand{\subparagraph}{
    \@ifstar
      \xxxSubParagraphStar
      \xxxSubParagraphNoStar
  }
  \newcommand{\xxxSubParagraphStar}[1]{\oldsubparagraph*{#1}\mbox{}}
  \newcommand{\xxxSubParagraphNoStar}[1]{\oldsubparagraph{#1}\mbox{}}
\fi
\makeatother


\providecommand{\tightlist}{%
  \setlength{\itemsep}{0pt}\setlength{\parskip}{0pt}}\usepackage{longtable,booktabs,array}
\usepackage{calc} % for calculating minipage widths
% Correct order of tables after \paragraph or \subparagraph
\usepackage{etoolbox}
\makeatletter
\patchcmd\longtable{\par}{\if@noskipsec\mbox{}\fi\par}{}{}
\makeatother
% Allow footnotes in longtable head/foot
\IfFileExists{footnotehyper.sty}{\usepackage{footnotehyper}}{\usepackage{footnote}}
\makesavenoteenv{longtable}
\usepackage{graphicx}
\makeatletter
\def\maxwidth{\ifdim\Gin@nat@width>\linewidth\linewidth\else\Gin@nat@width\fi}
\def\maxheight{\ifdim\Gin@nat@height>\textheight\textheight\else\Gin@nat@height\fi}
\makeatother
% Scale images if necessary, so that they will not overflow the page
% margins by default, and it is still possible to overwrite the defaults
% using explicit options in \includegraphics[width, height, ...]{}
\setkeys{Gin}{width=\maxwidth,height=\maxheight,keepaspectratio}
% Set default figure placement to htbp
\makeatletter
\def\fps@figure{htbp}
\makeatother
% definitions for citeproc citations
\NewDocumentCommand\citeproctext{}{}
\NewDocumentCommand\citeproc{mm}{%
  \begingroup\def\citeproctext{#2}\cite{#1}\endgroup}
\makeatletter
 % allow citations to break across lines
 \let\@cite@ofmt\@firstofone
 % avoid brackets around text for \cite:
 \def\@biblabel#1{}
 \def\@cite#1#2{{#1\if@tempswa , #2\fi}}
\makeatother
\newlength{\cslhangindent}
\setlength{\cslhangindent}{1.5em}
\newlength{\csllabelwidth}
\setlength{\csllabelwidth}{3em}
\newenvironment{CSLReferences}[2] % #1 hanging-indent, #2 entry-spacing
 {\begin{list}{}{%
  \setlength{\itemindent}{0pt}
  \setlength{\leftmargin}{0pt}
  \setlength{\parsep}{0pt}
  % turn on hanging indent if param 1 is 1
  \ifodd #1
   \setlength{\leftmargin}{\cslhangindent}
   \setlength{\itemindent}{-1\cslhangindent}
  \fi
  % set entry spacing
  \setlength{\itemsep}{#2\baselineskip}}}
 {\end{list}}
\usepackage{calc}
\newcommand{\CSLBlock}[1]{\hfill\break\parbox[t]{\linewidth}{\strut\ignorespaces#1\strut}}
\newcommand{\CSLLeftMargin}[1]{\parbox[t]{\csllabelwidth}{\strut#1\strut}}
\newcommand{\CSLRightInline}[1]{\parbox[t]{\linewidth - \csllabelwidth}{\strut#1\strut}}
\newcommand{\CSLIndent}[1]{\hspace{\cslhangindent}#1}

\usepackage{booktabs}
\usepackage{caption}
\usepackage{longtable}
\usepackage{colortbl}
\usepackage{array}
\usepackage{anyfontsize}
\usepackage{multirow}
\usepackage{wrapfig}
\usepackage{float}
\usepackage{pdflscape}
\usepackage{tabu}
\usepackage{threeparttable}
\usepackage{threeparttablex}
\usepackage[normalem]{ulem}
\usepackage{makecell}
\usepackage{xcolor}
\makeatletter
\@ifpackageloaded{bookmark}{}{\usepackage{bookmark}}
\makeatother
\makeatletter
\@ifpackageloaded{caption}{}{\usepackage{caption}}
\AtBeginDocument{%
\ifdefined\contentsname
  \renewcommand*\contentsname{Table of contents}
\else
  \newcommand\contentsname{Table of contents}
\fi
\ifdefined\listfigurename
  \renewcommand*\listfigurename{List of Figures}
\else
  \newcommand\listfigurename{List of Figures}
\fi
\ifdefined\listtablename
  \renewcommand*\listtablename{List of Tables}
\else
  \newcommand\listtablename{List of Tables}
\fi
\ifdefined\figurename
  \renewcommand*\figurename{Figure}
\else
  \newcommand\figurename{Figure}
\fi
\ifdefined\tablename
  \renewcommand*\tablename{Table}
\else
  \newcommand\tablename{Table}
\fi
}
\@ifpackageloaded{float}{}{\usepackage{float}}
\floatstyle{ruled}
\@ifundefined{c@chapter}{\newfloat{codelisting}{h}{lop}}{\newfloat{codelisting}{h}{lop}[chapter]}
\floatname{codelisting}{Listing}
\newcommand*\listoflistings{\listof{codelisting}{List of Listings}}
\makeatother
\makeatletter
\makeatother
\makeatletter
\@ifpackageloaded{caption}{}{\usepackage{caption}}
\@ifpackageloaded{subcaption}{}{\usepackage{subcaption}}
\makeatother

\ifLuaTeX
  \usepackage{selnolig}  % disable illegal ligatures
\fi
\usepackage{bookmark}

\IfFileExists{xurl.sty}{\usepackage{xurl}}{} % add URL line breaks if available
\urlstyle{same} % disable monospaced font for URLs
\hypersetup{
  pdftitle={Validation of the Spiritual Harm and Abuse Scale (SHAS) with the Rating Scale Model},
  pdfauthor={Jack Huber, Gonzaga University; Dan Koch, Northwest University},
  colorlinks=true,
  linkcolor={blue},
  filecolor={Maroon},
  citecolor={Blue},
  urlcolor={Blue},
  pdfcreator={LaTeX via pandoc}}


\title{Validation of the Spiritual Harm and Abuse Scale (SHAS) with the
Rating Scale Model}
\author{Jack Huber, Gonzaga University \and Dan Koch, Northwest
University}
\date{August 2024}

\begin{document}
\maketitle

\renewcommand*\contentsname{Table of contents}
{
\hypersetup{linkcolor=}
\setcounter{tocdepth}{2}
\tableofcontents
}

\bookmarksetup{startatroot}

\chapter*{Abstract}\label{abstract}
\addcontentsline{toc}{chapter}{Abstract}

\markboth{Abstract}{Abstract}

Spiritual harm and abuse has been a serious problem that has begun to
gain the attention of the clinical and psychological research community.
Some research has attempted to define spiritual harm and abuse and
documented lasting emotional and psychological damage to victims of
spiritual abuse. A valid scale of spiritual and religious abuse is
needed, and the nascent Spiritual Harm and Abuse Scale (SHAS) has
potential to meet this need. (Koch \& Edstrom, 2022) developed an item
pool and conducted an initial validation study of the SHAS which
established the multidimensional structure of the item pool. This study
builds on Koch \& Edstrom (2022) to further validate the SHAS by using
the Rasch Rating Scale Model.

\bookmarksetup{startatroot}

\chapter*{Introduction}\label{introduction}
\addcontentsline{toc}{chapter}{Introduction}

\markboth{Introduction}{Introduction}

Spiritual and religious abuse has been a serious problem which is
gaining the attention of the psychological research community.

Research to date has attempted to define spiritual abuse. Spiritual
abuse can include a wide range of psychologically controlling behaviors.
It can include diverse emotionally abusive behaviors such as rejecting,
isolating, terrorizing, ignoring, corrupting, verbally assaulting, and
over-pressuring (Pearl, 1994). It can include sexual or other physical
forms of abuse by a clergy member (Doyle, 2006). Spiritual abuse may
include theologically sanctioned mistreatment of a group of people such
LGBTQ+ persons (Foster et al., 2015). Spiritual abuse can range in
specificity from abusive acts by religious leaders to abuse perpetrated
by a religious group or person representing a religious group with a
religious or spiritual component (Swindle, 2017). Spiritual abuse can
also include quoting scripture or invoking divine rationale for
mistreatment (Lv. Oakley et al., 2018). Koch \& Edstrom (2022) define
spiritual abuse broadly as ``a type of emotional and psychological abuse
perpetrated by a religious leader or group and/or with a religious or
spiritual component, usually involving coercion or control.''

Spiritual and religious abuse has serious negative effects (L. R. Oakley
\& Kinmond, 2014; Stevens et al., 2019; Swindle, 2017; Ward, 2011).
Spiritual and religious abuse can be traumatic, and such trauma that
occurs in a religious or spiritual setting may deprive victims of
religion and spirituality as a coping strategy. Religious leaders may be
on a par with parents in their potential for abusive harm. There are
also concerns related to secondary trauma for clinicians working with
R/S abused clients (Gubi \& Jacobs, 2009).

In light of these concerns, a valid scale of spiritual harm and abuse is
needed. Practitioners who treat clients who may be victims of religious
and spiritual abuse need an instrument to assess the extent and severity
of spiritual abuse suffered by a client. Researchers need a reliable
measurement of spiritual abuse to examine relationships between
spiritual abuse and any number of other psychological, health, or other
measures.

One promising instrument is the nascent Spiritual Harm \& Abuse Scale
(SHAS) (Koch \& Edstrom, 2022). These authors drew from a recent
unpublished scale of religious and spiritual abuse (Keller, 2016), other
scales of abuse (Briere \& Runtz, 1989; Carlson et al., 2011; Kira et
al., 2008; Sanders \& Becker-Lausen, 1995), and interviews with victims
of abuse to document a wide range of experiences and emotional states
that could be described as spiritual harm and abuse. From this work they
developed a pool of 66 prompts for an online community survey which they
fielded in 2021. Exploratory factor analysis of the data revealed six
factors which the authors used to distill the instrument to a 27-item
scale with six subscales.

In this paper, we further validate the SHAS by using item response
theory. Item response theory can enhance clinical assessments in several
ways. First, IRT provides interval-level measures which are more
appropriate for statistical analysis than ordinal-level sum scores and
factor scores. Second, IRT provides one scale along which both items and
persons can be located. Third, measurements derived from IRT are
sample-free. This means it is possible to estimate a person's level of
the latent construct free of the distribution of individual items and to
estimate an item's difficulty level free from the distribution of people
used in the sample. Finally, IRT improves upon single overall
reliability coefficients like Cronbach's alpha by providing examination
of precision across the score continuum.

The goal of this study was thus to use the Rating Scale Model (Andrich,
1978) to validate the SHAS in several ways. First, we aimed to express
the SHAS as a single interval-level outcome measure of cumulative
lifetime severity of spiritual harm and abuse. Second, we aimed to
enrich the diagnostic value of the items and scale by examining
severity. Third, we aimed to examine the distribution of item severities
to see where the scale provides the most precise, reliable measurement
of spiritual abuse. Such a scale also lays the foundation to further
develop the SHAS by identifying anchor items that can calibrate with
pilot items on future forms. Finally, we aimed to establish a SHAS score
that is free of differential item functioning based on race, gender, and
other demographic variables.

\bookmarksetup{startatroot}

\chapter*{Methods}\label{methods}
\addcontentsline{toc}{chapter}{Methods}

\markboth{Methods}{Methods}

\section*{Item Pool Development}\label{item-pool-development}
\addcontentsline{toc}{section}{Item Pool Development}

\markright{Item Pool Development}

Koch \& Edstrom (2022) developed an initial pool of items to measure
spiritual harm and abuse. The authors validated the construct of
spiritual harm and abuse through literature review (Kvarfordt, 2010;
Nobakht \& Dale, 2017; L. R. Oakley \& Kinmond, 2014;
Rodríguez-Carballeira et al., 2015; Swindle, 2017; Ward, 2011; Winell,
n.d.) and interviews with survivors of religious and spiritual abuse.
These interviews revealed an array of specific examples of potential
abuse not described in the published literature such as abuse inflicted
by narcissistic persons, financial coercion, developmentally
inappropriate children's teachings, pressure to stay in physically
abusive marriages, neglect of needed medical care, sexual
discrimination, and shunning/shaming, among others. The authors
generated items by drawing from publicly available measures of similar
constructs and by writing new items to maximize construct coverage. Each
item they adapted to measure either a potentially spiritually abusive
event (inflicting emotional abuse by a religious leader or group and/or
with a religious or spiritual component, usually involving coercion or
control) or a theorized effect of spiritual abuse, based on the reports
of self-described survivors of spiritual abuse in the qualitative
literature. In developing items, the authors sought to balance construct
coverage with alternate wording and efficiency for participants. They
wanted to cover as much of the domain as possible in a survey that could
be completed in 15 minutes.

\section*{Participants and
Procedures}\label{participants-and-procedures}
\addcontentsline{toc}{section}{Participants and Procedures}

\markright{Participants and Procedures}

Participants in the initial SHAS validation study were adults who
responded to an online survey. To qualify for the survey, participants
had to be at least 18 years of age and to have identified themselves as
Christian in the past (Koch \& Edstrom, 2022).

The second author used snowball sampling to recruit participants to the
study. He promoted the survey through various podcast feeds (in audio
form during a podcast episode), e-mail lists, Facebook groups, and other
online groups associated with progressive Christian podcasts based in
the United States. Listeners/readers were encouraged to invite others
who satisfied the criteria to take part in the survey. The second author
also posted the link on his own social media accounts with a similar
encouragement to invite others who satisfied the study criteria. The
sampling priority was to reach the largest, most inclusive, community
sample. Participants were consented before beginning the survey. The
study was approved by the IRB at Northwest University. Data collection
began on January 28, 2021 and concluded on February 27, 2021.

\section*{Measures}\label{measures}
\addcontentsline{toc}{section}{Measures}

\markright{Measures}

\subsection*{Demographics}\label{demographics}
\addcontentsline{toc}{subsection}{Demographics}

Demographic variables included age, gender, race, and sexual
orientation.

\subsection*{Religious Background
Characteristics}\label{religious-background-characteristics}
\addcontentsline{toc}{subsection}{Religious Background Characteristics}

The questionnaire included several items asking about religious matters.
These included items asking whether the respondent was raised in a
Christian home, denomination (and demographics of membership and
leadership) in which the respondent experienced abuse, current weekly
consumption of progressive Christian podcast episodes, current religious
identification, current theological identification, and current view of
the Bible.

\subsection*{Spiritual Harm and Abuse
Items}\label{spiritual-harm-and-abuse-items}
\addcontentsline{toc}{subsection}{Spiritual Harm and Abuse Items}

The final survey instrument included a total of 66 prompts separated
into two distinct categories (Koch \& Edstrom, 2022). The first section,
External Events, included 52 prompts about potentially abusive
experiences. Prompts in this section measured prevalence by asking ``How
often have you experienced the above in a church or Christian group
setting?'' and providing participants a Liker scale of five response
categories: (1) Never, (2) Once or twice, (3) Sometimes, (4) Often, and
(5) All the time. The large number of prompts in this first section were
presented in a randomized order so that all prompts would receive
responses by approximately the same number of participants even if many
participants did not complete all of the prompts in this section.

The second section, Internal States, included 14 prompts about personal
feelings resulting from the abusive experiences described in the first
section. Prompts in this section measured prevalence by asking ``At any
point, how often have you experienced the above as a result of negative
religious experiences?'' Participants were provided the same 5-point
Likert scale of response categories as in the first section: (1) Never,
(2) Once or twice, (3) Sometimes, (4) Often, and (5) All the time.

\section*{Data Analysis}\label{data-analysis}
\addcontentsline{toc}{section}{Data Analysis}

\markright{Data Analysis}

In their initial validation of the SHAS, Koch \& Edstrom (2022) examined
the dimensionality of the 66 items. The authors used exploratory factor
analysis (EFA) to determine the factor structure of the items. In these
EFA analyses, the items loaded onto six inter-related factors of
eigenvalues of at least 1. This enabled the authors to distill the item
pool from 66 to 27 items and to identify several meaningful subscales
for potential future use by clinicians and researchers.

For this study, we focused on the commonality of the items in order to
use item response theory to establish a single interval scale of
spiritual abuse. Item response theory models make two assumptions: (1) a
single latent trait explains responses to items (unidimensionality) and
(2) after controlling for this latent trait, items are weakly correlated
(local independence) (Embretson \& Reise, 2000; Reise, Bonifay, et al.,
2013; Reise \& Waller, 2009). The SHAS items were multidimensional by
design. The construct of spiritual harm and abuse is not conceptually
narrow, and to cover the breadth of this construct, Koch \& Edstrom
(2022) wrote two distinct item sets. However, a unidimensional IRT model
can be appropriately fit to multidimensional data (Reise, Moore, et al.,
2013). In this study we hypothesized a general factor of spiritual abuse
as the primary cause of responses to the full set of 66 SHAS items. We
therefore sought evidence of unidimensionality and local independence in
order to calibrate the items and participants to a single IRT scale.

We began by examining classical item statistics to identify poorly
functioning items based on extreme values of skew and kurtosis and low
item-total correlations. Then we conducted several analyses of
dimensionality in search of sufficient evidence of unidimensionality and
local independence. This included principal components analysis to
compare the eigenvalues of the largest and second largest components.
Like factor analysis, component analysis attempts to summarize the
common variance in items by identifying a smaller set of latent sources
of covariation. Principal components analysis differs from factor
analysis by treating all of the common variance among the items as the
total variance to be explained {[}verify this{]}. We estimated a
unidimensional Rasch model and calculated the proportion of variance in
the data it explained. This we followed with a Principal Components
Analysis of Residuals (PCAR) to gauge the importance of components in
the item residuals, and with a similar statistic, Q3 ({``Effects of
Local Item Dependence on the Fit and Equating Performance of the
Three-Parameter Logistic Model,''} 1984), to further inspect the
correlations among item residuals.

We estimated a Rating Scale model (Andrich, 1978) to the SHAS items. We
began by examining plots of the category characteristic curves of the
items to visualize step parameters. Then we turned to item severity
parameters and item fit statistics to inspect the technical quality of
the items. We examined an item-person map to understand the fit of the
items to the participants. We concluded by assessing the fit of the
model to the data.

Finally, we conducted differential item functioning (DIF) analysis of
the items to examine item bias by gender, and age, and race. {[}We
applied logistic ordinal regression with IRT scoring. We used the
Chi-squared likelihood-ratio statistic as the initial DIF detection
criteria (alpha \textless{} 0.01), and a cut-off of McFadden pseudo R2Δ
≥ 0.02 in model comparisons to determine substantial DIF, a reasonable
threshold used in the development of self-reported health outcomes.{]}

We conducted all analyses in \textbf{R} (R Core Team, 2024). We used the
\textbf{psych} (Revelle, 2024) and \textbf{eRm} (Mair et al., 2024)
packages for analyses of dimensionality, the \textbf{mirt} (Chalmers,
2024) and \textbf{TAM} (Robitzsch et al., 2024) packages for Rating
Scale analyses, and the \textbf{lordif} package for DIF analyses.

\bookmarksetup{startatroot}

\chapter*{Results}\label{results}
\addcontentsline{toc}{chapter}{Results}

\markboth{Results}{Results}

\section*{Participant
Characteristics}\label{participant-characteristics}
\addcontentsline{toc}{section}{Participant Characteristics}

\markright{Participant Characteristics}

In total, 3,222 individuals responded to the survey. Participants who
did not opt out and completed at least 50\% of the spiritual abuse item
pool were considered complete responders and included in the study (N =
3,219). Non-responders included those who responded to less than half of
the items (N=3).

Most of the participants responded to the demographics items, and we
present the results of these items in Table~\ref{tbl-DemsTable}. Most of
the sample (76\%) was adults between the ages of 25 and 56, with
Millennials (Ages 25-40) comprising nearly 56\% of the sample, followed
by Generation X (Ages 41-56) comprising 21\%. Half the sample was
cisgender female (51\%) and 41\% was cisgender male. The sample was also
predominantly white (87\%) and straight (82\%).

\begingroup
\fontsize{12.0pt}{14.4pt}\selectfont
\setlength{\LTpost}{0mm}

\begin{longtable}{lc}

\caption{\label{tbl-DemsTable}Demographic Characteristics}

\tabularnewline

\toprule
\textbf{Characteristic} & \textbf{N = 3,222}\textsuperscript{\textit{1}} \\ 
\midrule\addlinespace[2.5pt]
{\bfseries Age} & 36 (30, 44) \\ 
    Unknown & 166 \\ 
{\bfseries Generation} &  \\ 
    2 - Generation (Ages 24-39) & 1,770 (58\%) \\ 
    3 - Generation (Ages 40-55) & 743 (24\%) \\ 
    4 - Generation (Ages 56-74) & 341 (11\%) \\ 
    1 - Generation (Below age 24) & 182 (6.0\%) \\ 
    5 - Generation (Above age 74) & 20 (0.7\%) \\ 
    Unknown & 166 \\ 
{\bfseries Gender} &  \\ 
    Cisgender female & 1,635 (53\%) \\ 
    Cisgender male & 1,320 (43\%) \\ 
    Other & 49 (1.6\%) \\ 
    Prefer not to say & 36 (1.2\%) \\ 
    Transgender female & 12 (0.4\%) \\ 
    Transgender male & 8 (0.3\%) \\ 
    Unknown & 162 \\ 
{\bfseries Race} &  \\ 
    White/European American & 2,786 (91\%) \\ 
    Other & 92 (3.0\%) \\ 
    Ethnically of Hispanic/Latino Origin & 73 (2.4\%) \\ 
    Black/African American & 51 (1.7\%) \\ 
    Asian & 47 (1.5\%) \\ 
    American Indian/Alaska Native & 7 (0.2\%) \\ 
    Native Hawaiian/Pacific Islander & 4 (0.1\%) \\ 
    Unknown & 162 \\ 
{\bfseries Sexual orientation} &  \\ 
    Straight & 2,626 (86\%) \\ 
    Bisexual & 174 (5.7\%) \\ 
    Gay/Lesbian & 102 (3.3\%) \\ 
    Queer & 67 (2.2\%) \\ 
    Don't know & 51 (1.7\%) \\ 
    Asexual & 44 (1.4\%) \\ 
    Unknown & 158 \\ 
{\bfseries Podcast Use} &  \\ 
    1 - One episode per week & 1,460 (48\%) \\ 
    2 - Two episodes per week & 1,349 (44\%) \\ 
    3 - Three or more episodes per week & 251 (8.2\%) \\ 
    Unknown & 162 \\ 
\bottomrule

\end{longtable}

\begin{minipage}{\linewidth}
\textsuperscript{\textit{1}}Median (Q1, Q3); n (\%)\\
\end{minipage}
\endgroup

The survey also included several items asking participants about their
religious characteristics. We report the results of these religious
characteristics in Table~\ref{tbl-ReligTable}. Most of the sample (87\%)
was raised in a Christian home. Over half (56\%) now self-identify as
Protestant, followed by a quarter who do not identify with a religious
tradition (16\% ``Nothing in particular'' and 10\% ``Agnostic''). Over
60\% view the Bible as divinely inspired if not to be interpreted
literally.

\begingroup
\fontsize{12.0pt}{14.4pt}\selectfont
\setlength{\LTpost}{0mm}

\begin{longtable}{lc}

\caption{\label{tbl-ReligTable}Religious Characteristics}

\tabularnewline

\toprule
\textbf{Characteristic} & \textbf{N = 3,222}\textsuperscript{\textit{1}} \\ 
\midrule\addlinespace[2.5pt]
{\bfseries Raised in Christian home} & 2,795 (91\%) \\ 
    Unknown & 161 \\ 
{\bfseries Denomination of most abuse} &  \\ 
    Non-denominational & 751 (25\%) \\ 
    Baptist & 585 (19\%) \\ 
    Charismatic/Pentecostal & 448 (15\%) \\ 
    Mainline & 316 (10\%) \\ 
    Christian other than above & 301 (9.9\%) \\ 
    Mormon & 130 (4.3\%) \\ 
    Christian school or university & 124 (4.1\%) \\ 
    Catholic & 118 (3.9\%) \\ 
    Evangelistic/Missions organization & 110 (3.6\%) \\ 
    College Christian organization & 55 (1.8\%) \\ 
    Christian group other than the above & 49 (1.6\%) \\ 
    Local Bible study or small group & 23 (0.8\%) \\ 
    Christian Youth Organization & 15 (0.5\%) \\ 
    Orthodox Christian & 11 (0.4\%) \\ 
    Christian media company & 8 (0.3\%) \\ 
    Christian political/activist group & 3 (<0.1\%) \\ 
    Christian aid/social services group & 0 (0\%) \\ 
    Unknown & 175 \\ 
{\bfseries Racial makeup of abusive community} &  \\ 
    Predominantly white & 2,632 (87\%) \\ 
    Mixed race & 356 (12\%) \\ 
    Predominantly people of color & 54 (1.8\%) \\ 
    Unknown & 180 \\ 
{\bfseries Racial makeup of abusive group leadership} &  \\ 
    Predominantly white & 2,807 (92\%) \\ 
    Mixed race & 184 (6.0\%) \\ 
    Predominantly people of color & 53 (1.7\%) \\ 
    Unknown & 178 \\ 
{\bfseries Gender makeup of abusive group leadership} &  \\ 
    All male & 1,253 (41\%) \\ 
    Mostly male & 1,129 (37\%) \\ 
    Mixed gender & 639 (21\%) \\ 
    Mostly female & 24 (0.8\%) \\ 
    All female & 3 (<0.1\%) \\ 
    Unknown & 174 \\ 
{\bfseries Involvement in abusive group} &  \\ 
    7 - Strongly agree & 1,529 (50\%) \\ 
    6 & 503 (16\%) \\ 
    5 & 328 (11\%) \\ 
    1 - Strongly disagree & 290 (9.5\%) \\ 
    4 & 174 (5.7\%) \\ 
    2 & 121 (4.0\%) \\ 
    3 & 109 (3.6\%) \\ 
    Unknown & 168 \\ 
{\bfseries Current religious identification} &  \\ 
    Protestant & 1,833 (60\%) \\ 
    Nothing in particular & 518 (17\%) \\ 
    Agnostic & 325 (11\%) \\ 
    Atheist & 128 (4.2\%) \\ 
    Orthodox Christian & 79 (2.6\%) \\ 
    Catholic & 66 (2.2\%) \\ 
    Mormon & 57 (1.9\%) \\ 
    Other religious tradition & 51 (1.7\%) \\ 
    Unknown & 165 \\ 
{\bfseries Current theological identification} &  \\ 
    5 & 477 (23\%) \\ 
    4 - In the midddle & 468 (23\%) \\ 
    6 & 384 (19\%) \\ 
    3 & 263 (13\%) \\ 
    7 - Very liberal & 191 (9.4\%) \\ 
    2 & 175 (8.6\%) \\ 
    1 - Very conservative & 72 (3.5\%) \\ 
    Unknown & 1,192 \\ 
{\bfseries Current view of Bible} &  \\ 
    The Bible is the inspired word of God but not everything should be taken literally, word for word & 2,010 (66\%) \\ 
    The Bible is an ancient book of fables, legends, history and moral precepts recorded by man & 870 (28\%) \\ 
    The Bible is the actual word of God and is to be taken literally, word for word & 174 (5.7\%) \\ 
    Unknown & 168 \\ 
\bottomrule

\end{longtable}

\begin{minipage}{\linewidth}
\textsuperscript{\textit{1}}n (\%)\\
\end{minipage}
\endgroup

\section*{Initial Item Analysis}\label{initial-item-analysis}
\addcontentsline{toc}{section}{Initial Item Analysis}

\markright{Initial Item Analysis}

In Table~\ref{tbl-ClassicalItemTable} we report classical statistics for
the 66 SHAS items. We sorted the items by their mean. We flagged six
items due to extreme values of skew (less than -2 or greater than +2) or
kurtosis (less than -7 or greater than +7) and item-total correlations
below 0.50. These included items EQ17, EQ27, EQ41, EQ46, EQ55, and EQ60.
Category response proportions show that most people responded ``Never''
to these items. Thus, because these items do not apply to the vast
majority of people in this population, we removed them from the item
pool. This reduced the pool from 66 to 59 items.

\begin{table}

\caption{\label{tbl-ClassicalItemTable}Classical Item Statistics}

\centering{

\begin{tabu} to \linewidth {>{\raggedright}X>{\raggedleft}X>{\raggedleft}X>{\raggedleft}X>{\raggedleft}X>{\raggedleft}X>{\raggedleft}X>{\raggedleft}X>{\raggedleft}X>{\raggedleft}X>{\raggedleft}X>{\raggedleft}X>{\raggedright}X}
\hline
  & 1 & 2 & 3 & 4 & 5 & n & mean & sd & skew & kurtosis & Item-total correlation & flag\\
\hline
EQ46 - Treated as less than because of my skin color & 0.94 & 0.03 & 0.02 & 0.01 & 0.00 & 3,211 & 1.11 & 0.48 & 5.15 & 28.93 & 0.16 & *\\
\hline
EQ41 - Denied opportunities to serve because of my sexual orientation & 0.89 & 0.02 & 0.03 & 0.02 & 0.03 & 3,211 & 1.27 & 0.88 & 3.30 & 9.87 & 0.31 & *\\
\hline
EQ17 - Treated as less than b/c my sexual orientation & 0.88 & 0.02 & 0.03 & 0.03 & 0.04 & 3,208 & 1.34 & 0.97 & 2.87 & 6.96 & 0.35 & *\\
\hline
EQ47 - Medical care being postponed/withheld for religious reasons & 0.70 & 0.17 & 0.09 & 0.02 & 0.01 & 3,218 & 1.48 & 0.86 & 1.93 & 3.40 & 0.52 & \\
\hline
EQ60 - Encouraged by leader to stay in abusive marriage & 0.76 & 0.09 & 0.07 & 0.05 & 0.02 & 3,214 & 1.49 & 0.99 & 2.05 & 3.22 & 0.48 & *\\
\hline
EQ55 - Hearing cultural references in sermons unfamiliar to my race/eth. subculture & 0.66 & 0.18 & 0.11 & 0.04 & 0.02 & 3,205 & 1.58 & 0.95 & 1.66 & 2.12 & 0.28 & *\\
\hline
IQ72 - Feeling as if God harmed me directly & 0.64 & 0.17 & 0.13 & 0.04 & 0.02 & 3,070 & 1.64 & 0.99 & 1.52 & 1.55 & 0.54 & \\
\hline
IQ81 - Nightmares about my negative religious experiences & 0.56 & 0.21 & 0.15 & 0.06 & 0.02 & 3,073 & 1.77 & 1.04 & 1.23 & 0.66 & 0.65 & \\
\hline
EQ65 - Cut off/shunned by more religious family members & 0.59 & 0.16 & 0.13 & 0.08 & 0.04 & 3,220 & 1.82 & 1.18 & 1.27 & 0.48 & 0.61 & \\
\hline
EQ10 - Asked to give up personal goals by pastor & 0.59 & 0.16 & 0.13 & 0.08 & 0.04 & 3,220 & 1.83 & 1.18 & 1.25 & 0.44 & 0.64 & \\
\hline
EQ26 - Pressured to forgive abuser while abuse was ongoing & 0.60 & 0.14 & 0.11 & 0.09 & 0.06 & 3,212 & 1.87 & 1.27 & 1.24 & 0.21 & 0.66 & \\
\hline
EQ24 - Expected to consult pastor/leader before making non-religious decisions & 0.56 & 0.16 & 0.16 & 0.08 & 0.04 & 3,221 & 1.89 & 1.18 & 1.10 & 0.07 & 0.66 & \\
\hline
EQ23 - Prayer replacing needed medical interventions & 0.51 & 0.21 & 0.17 & 0.08 & 0.03 & 3,219 & 1.92 & 1.13 & 1.02 & 0.02 & 0.64 & \\
\hline
EQ27 - Denied opportunities to serve b/c of my gender & 0.60 & 0.11 & 0.12 & 0.09 & 0.08 & 3,217 & 1.94 & 1.35 & 1.14 & -0.13 & 0.48 & *\\
\hline
EQ31 - Shunned/ignored by pastor/church/group & 0.46 & 0.22 & 0.18 & 0.09 & 0.05 & 3,217 & 2.03 & 1.19 & 0.92 & -0.20 & 0.64 & \\
\hline
EQ30 - Deterred from seeking mental health treatment/counseling/medication & 0.50 & 0.17 & 0.17 & 0.11 & 0.05 & 3,219 & 2.04 & 1.25 & 0.88 & -0.46 & 0.70 & \\
\hline
EQ43 - Shamed by pastor/group for poor spiritual/moral performance & 0.46 & 0.22 & 0.18 & 0.10 & 0.04 & 3,217 & 2.04 & 1.18 & 0.86 & -0.31 & 0.73 & \\
\hline
IQ68 - Anxiety attacks triggered by religious stimuli & 0.49 & 0.17 & 0.19 & 0.11 & 0.05 & 3,073 & 2.06 & 1.23 & 0.83 & -0.51 & 0.68 & \\
\hline
EQ52 - Blamed for harm I suffered rather than blaming who harmed me & 0.48 & 0.19 & 0.16 & 0.12 & 0.05 & 3,217 & 2.08 & 1.26 & 0.85 & -0.52 & 0.76 & \\
\hline
IQ70 - Feeling betrayed by God & 0.42 & 0.24 & 0.21 & 0.09 & 0.05 & 3,069 & 2.11 & 1.18 & 0.79 & -0.36 & 0.56 & \\
\hline
EQ49 - Leadership/group protecting and elevating abusive individuals & 0.44 & 0.22 & 0.18 & 0.11 & 0.05 & 3,213 & 2.12 & 1.23 & 0.79 & -0.50 & 0.72 & \\
\hline
EQ16 - Church/community abandoning me in difficult time & 0.45 & 0.21 & 0.17 & 0.11 & 0.06 & 3,217 & 2.13 & 1.27 & 0.83 & -0.50 & 0.69 & \\
\hline
EQ57 - Developing mental/physical ailments from conforming to group/leader’s expectation & 0.48 & 0.16 & 0.16 & 0.12 & 0.07 & 3,219 & 2.15 & 1.33 & 0.78 & -0.70 & 0.72 & \\
\hline
EQ20 - Members pressured to give money despite financial hardship & 0.50 & 0.15 & 0.15 & 0.11 & 0.09 & 3,217 & 2.16 & 1.39 & 0.82 & -0.71 & 0.68 & \\
\hline
EQ62 - Scripture used to justify physical violence & 0.43 & 0.19 & 0.21 & 0.11 & 0.05 & 3,220 & 2.16 & 1.24 & 0.69 & -0.67 & 0.65 & \\
\hline
EQ59 - Pastor/group blame victim for their abuse & 0.42 & 0.21 & 0.19 & 0.12 & 0.05 & 3,217 & 2.16 & 1.23 & 0.70 & -0.65 & 0.76 & \\
\hline
EQ21 - Taught I would risk Hell if left my church/group & 0.51 & 0.13 & 0.13 & 0.10 & 0.13 & 3,218 & 2.20 & 1.47 & 0.81 & -0.83 & 0.69 & \\
\hline
EQ50 - Treated as less than because of my gender & 0.52 & 0.08 & 0.15 & 0.13 & 0.11 & 3,219 & 2.23 & 1.47 & 0.70 & -1.04 & 0.54 & \\
\hline
IQ82 - Having trouble navigating life outside my church/community & 0.38 & 0.23 & 0.24 & 0.11 & 0.05 & 3,071 & 2.24 & 1.21 & 0.62 & -0.62 & 0.61 & \\
\hline
EQ64 - Witnessing women pressured to stay in unfaithful/abusive marriages & 0.39 & 0.23 & 0.19 & 0.12 & 0.07 & 3,220 & 2.26 & 1.27 & 0.67 & -0.70 & 0.71 & \\
\hline
EQ28 - Leadershi/group protecting abusive individuals & 0.37 & 0.24 & 0.19 & 0.13 & 0.06 & 3,216 & 2.26 & 1.25 & 0.63 & -0.73 & 0.72 & \\
\hline
EQ40 - Threatening Divine punishment to keep group members in line & 0.44 & 0.17 & 0.18 & 0.12 & 0.10 & 3,220 & 2.28 & 1.39 & 0.67 & -0.88 & 0.75 & \\
\hline
EQ63 - Shamed by pastor/group for raising questions or concerns & 0.36 & 0.23 & 0.22 & 0.13 & 0.06 & 3,217 & 2.30 & 1.25 & 0.58 & -0.77 & 0.77 & \\
\hline
EQ19 - Behavior excessively monitored by pastor/group & 0.40 & 0.18 & 0.19 & 0.15 & 0.08 & 3,219 & 2.32 & 1.33 & 0.56 & -0.96 & 0.75 & \\
\hline
EQ51 - Extreme pressure to be pastor/missionary/spiritual leader & 0.41 & 0.19 & 0.17 & 0.14 & 0.10 & 3,216 & 2.33 & 1.38 & 0.59 & -0.98 & 0.62 & \\
\hline
EQ53 - Scripture used to justify abusive parent-child behavior & 0.39 & 0.19 & 0.20 & 0.14 & 0.09 & 3,218 & 2.36 & 1.35 & 0.55 & -0.97 & 0.73 & \\
\hline
IQ79 - Distrust of God & 0.33 & 0.23 & 0.24 & 0.12 & 0.08 & 3,068 & 2.40 & 1.27 & 0.50 & -0.81 & 0.61 & \\
\hline
EQ34 - Scripture used to justify physical punishment/severe discipline & 0.34 & 0.19 & 0.22 & 0.16 & 0.09 & 3,219 & 2.46 & 1.33 & 0.42 & -1.05 & 0.65 & \\
\hline
EQ29 - Feeling special when in pastor’s good graces; otherwise ignored & 0.34 & 0.19 & 0.22 & 0.16 & 0.09 & 3,213 & 2.47 & 1.34 & 0.40 & -1.08 & 0.69 & \\
\hline
IQ73 - Self-hatred or self-loathing & 0.31 & 0.21 & 0.24 & 0.15 & 0.09 & 3,072 & 2.50 & 1.31 & 0.38 & -1.00 & 0.64 & \\
\hline
EQ11 - Disagree w/pastor portrayed as evil & 0.32 & 0.21 & 0.20 & 0.17 & 0.10 & 3,216 & 2.52 & 1.35 & 0.38 & -1.12 & 0.77 & \\
\hline
EQ58 - Terror/horror used to motivate religious decisions & 0.32 & 0.18 & 0.24 & 0.16 & 0.10 & 3,220 & 2.53 & 1.34 & 0.34 & -1.09 & 0.73 & \\
\hline
EQ7 - Pastor speaking on God's behalf & 0.35 & 0.18 & 0.19 & 0.16 & 0.13 & 3,215 & 2.54 & 1.43 & 0.39 & -1.21 & 0.67 & \\
\hline
EQ42 - Seeing others shamed/shunned by pastor/leader/group & 0.25 & 0.24 & 0.26 & 0.17 & 0.08 & 3,215 & 2.60 & 1.25 & 0.28 & -0.96 & 0.77 & \\
\hline
EQ44 - Love/acceptance offered only for high spiritual/moral performance & 0.30 & 0.18 & 0.23 & 0.19 & 0.10 & 3,217 & 2.61 & 1.35 & 0.24 & -1.20 & 0.80 & \\
\hline
EQ56 - Being made to feel I was crazy/weird for doubts/questions & 0.27 & 0.17 & 0.23 & 0.21 & 0.12 & 3,217 & 2.72 & 1.37 & 0.14 & -1.24 & 0.82 & \\
\hline
EQ18 - Church/pastor discourage critical thinking & 0.27 & 0.17 & 0.24 & 0.19 & 0.13 & 3,220 & 2.75 & 1.37 & 0.14 & -1.21 & 0.77 & \\
\hline
EQ45 - Developmentally-inappropriate/anxious Hell/Satan/demons taught to young children & 0.29 & 0.17 & 0.20 & 0.18 & 0.16 & 3,219 & 2.76 & 1.45 & 0.18 & -1.33 & 0.72 & \\
\hline
IQ78 - A lack of self-worth & 0.23 & 0.19 & 0.27 & 0.21 & 0.11 & 3,071 & 2.77 & 1.30 & 0.10 & -1.09 & 0.67 & \\
\hline
EQ36 - Mental/physical problems interpreted as spiritual/moral weakness & 0.24 & 0.19 & 0.24 & 0.21 & 0.12 & 3,216 & 2.78 & 1.34 & 0.11 & -1.17 & 0.77 & \\
\hline
EQ61 - Developmentally-inappropriate/anxious end times descriptions taught to young children & 0.30 & 0.15 & 0.20 & 0.19 & 0.17 & 3,219 & 2.78 & 1.47 & 0.14 & -1.37 & 0.72 & \\
\hline
EQ32 - Feeling unable to express unhappiness & 0.24 & 0.16 & 0.27 & 0.21 & 0.11 & 3,219 & 2.79 & 1.32 & 0.05 & -1.14 & 0.73 & \\
\hline
IQ74 - Sadness over the loss of my faith/religious community & 0.22 & 0.19 & 0.27 & 0.21 & 0.11 & 3,071 & 2.79 & 1.29 & 0.07 & -1.08 & 0.60 & \\
\hline
IQ76 - Feeling I wasted years of my life in a church/set of beliefs & 0.28 & 0.16 & 0.19 & 0.18 & 0.19 & 3,072 & 2.83 & 1.48 & 0.12 & -1.39 & 0.74 & \\
\hline
EQ54 - Being explicitly taught to distrust my intuitions & 0.23 & 0.16 & 0.24 & 0.20 & 0.17 & 3,217 & 2.92 & 1.40 & 0.01 & -1.26 & 0.75 & \\
\hline
EQ37 - Made to feel less spiritually mature than pastor/leadership & 0.22 & 0.17 & 0.24 & 0.21 & 0.16 & 3,221 & 2.94 & 1.38 & -0.01 & -1.22 & 0.75 & \\
\hline
EQ38 - Unrealistic demands placed on my moral/religious behavior & 0.22 & 0.16 & 0.24 & 0.20 & 0.18 & 3,217 & 2.95 & 1.40 & -0.01 & -1.26 & 0.80 & \\
\hline
IQ71 - Avoiding religious activities/settings to reduce distressing feelings & 0.19 & 0.17 & 0.24 & 0.22 & 0.17 & 3,074 & 3.01 & 1.36 & -0.06 & -1.19 & 0.74 & \\
\hline
IQ69 - Lack of spiritual direction/purpose & 0.13 & 0.20 & 0.32 & 0.23 & 0.12 & 3,071 & 3.01 & 1.20 & -0.06 & -0.83 & 0.57 & \\
\hline
EQ39 - Made to feel shame over natural sexual desires (not actions) & 0.24 & 0.13 & 0.20 & 0.21 & 0.22 & 3,219 & 3.05 & 1.48 & -0.12 & -1.37 & 0.72 & \\
\hline
IQ80 - Feeling isolated & 0.14 & 0.18 & 0.29 & 0.27 & 0.13 & 3,070 & 3.07 & 1.23 & -0.17 & -0.91 & 0.68 & \\
\hline
IQ77 - Anger upon reflecting on negative religious experiences & 0.15 & 0.18 & 0.26 & 0.26 & 0.15 & 3,075 & 3.08 & 1.28 & -0.16 & -1.02 & 0.77 & \\
\hline
EQ35 - Taught to distrust my emotions & 0.19 & 0.14 & 0.23 & 0.23 & 0.21 & 3,217 & 3.12 & 1.39 & -0.18 & -1.20 & 0.74 & \\
\hline
EQ25 - Feeling unable to raise questions and issues & 0.15 & 0.15 & 0.25 & 0.27 & 0.18 & 3,217 & 3.18 & 1.32 & -0.26 & -1.04 & 0.76 & \\
\hline
EQ66 - Others treated as less than due to their sexual orientation & 0.18 & 0.12 & 0.21 & 0.25 & 0.24 & 3,216 & 3.27 & 1.41 & -0.34 & -1.15 & 0.71 & \\
\hline
EQ22 - Expected to follow pastor/leader rules re: dating/marriage/sex & 0.20 & 0.11 & 0.15 & 0.24 & 0.30 & 3,218 & 3.33 & 1.49 & -0.39 & -1.28 & 0.70 & \\
\hline
\end{tabu}

}

\end{table}%

\section*{Unidimensionality and Local
Independence}\label{unidimensionality-and-local-independence}
\addcontentsline{toc}{section}{Unidimensionality and Local Independence}

\markright{Unidimensionality and Local Independence}

\subsection*{Principal Components}\label{principal-components}
\addcontentsline{toc}{subsection}{Principal Components}

We first analyzed the principal components in the 59 SHAS items. As
shown in Figure~\ref{fig-pca}, the first component was large, accounting
for nearly 30\% of the variance. The second, third, and fourth
components accounted for 4\%, 3\%, and 2\% of the variance,
respectively. The ratio of the eigenvalue of the first (largest)
component to that of the second is approximately 13.

\begin{figure}

\centering{

\includegraphics{results_files/figure-pdf/fig-pca-1.pdf}

}

\caption{\label{fig-pca}Principal Components}

\end{figure}%

\subsection*{Proportion of Variance}\label{proportion-of-variance}
\addcontentsline{toc}{subsection}{Proportion of Variance}

Second, we estimated a unidimensional Rasch model on dichotomized
responses to the 59 SHAS items and saved the residuals of the person
parameters. We calculated the variance in the observed item responses
and the variance of the residuals. Reckase (1979) suggests that the
unidimensionality assumption is safely met if the Rasch model explains
20\% of the variance in the data. In this case, the proportion of
variance in the SHAS item data explained by the Rasch model was .23.
Thus, the SHAS data meet this criterion.

\subsection*{Principal Component Analysis of Residuals
(PCAR)}\label{principal-component-analysis-of-residuals-pcar}
\addcontentsline{toc}{subsection}{Principal Component Analysis of
Residuals (PCAR)}

Next, we examined the principal components of the correlations among
residuals of the Rasch analysis. The premise is that once the Rasch
model has been estimated, correlations among the item residuals should
be minimal. Linacre () suggests that contrasts with eigenvalues of 2.0
or below can be considered noise. In Figure~\ref{fig-pcar}, our PCAR
analysis found only one contrast that rose above that 2.0 threshold.

\begin{figure}

\centering{

\includegraphics{results_files/figure-pdf/fig-pcar-1.pdf}

}

\caption{\label{fig-pcar}Contrasts from PCA of Standardized Residual
Correlations}

\end{figure}%

\subsection*{Q3}\label{q3}
\addcontentsline{toc}{subsection}{Q3}

Finally, we calculated the Q3 statistic {``Effects of Local Item
Dependence on the Fit and Equating Performance of the Three-Parameter
Logistic Model''} (1984) to examine correlations among item residuals,
the premise being that the latent trait should account for so much
common variance in the item responses that any net correlations among
the items should be weak. The Q3 statistic index criteria are that the
raw residual correlation between pairs of items should never exceed 0.10
(Marais \& Andrich, 2008). Of the 1,711 item pairs, the mean correlation
was -0.017 with a standard deviation of 0.035. Thus, most of the
residual correlations among items were very weak. Figure~\ref{fig-q3}
plots the matrix of correlations between item residuals. Each square
depicts a correlation. The squares are shaded in grey with the lightest
shade indicating the weakest correlations. As hoped, the vast majority
of squares are very light, indicating weak correlations among the item
residuals.

\begin{verbatim}

WLE Reliability= 0.918 
Yen's Q3 Statistic based on an estimated theta score 
*** 60 Items | 1770 item pairs
*** Q3 Descriptives
     M     SD    Min    10%    25%    50%    75%    90%    Max 
-0.017  0.036 -0.109 -0.055 -0.040 -0.020 -0.001  0.025  0.321 
\end{verbatim}

\begin{figure}

\centering{

\includegraphics{results_files/figure-pdf/fig-q3-1.pdf}

}

\caption{\label{fig-q3}Correlation Matrix of Item Residuals}

\end{figure}%

\section*{Rasch Rating Scale
Analysis}\label{rasch-rating-scale-analysis}
\addcontentsline{toc}{section}{Rasch Rating Scale Analysis}

\markright{Rasch Rating Scale Analysis}

We used the Rasch Rating Scale Model (RSM) (Andrich, 1978) to examine
the SHAS item pool. The RSM is a Rasch model (Rasch, 1960) that combines
(or ``calibrates'') information from items with information from persons
to arrive at a common scale for measuring both an item's and a person's
level of a latent trait, in this case, severity of spiritual abuse. This
scale is expressed in logits (log odds units). Whereas the Rasch model
was designed for use with dichotomous items (0 = incorrect response, 1 =
correct response), the RSM is an extension for use with polytomous items
based on rating scales such as Likert scales of agreement (Embretson \&
Reise, 2000).

We also use different language from conventional IRT analyses. Because
IRT was developed to measure student achievement, IRT statistics for
both items and persons tend to use language of ``ability'' and
``difficulty.'' For our purpose to measure a psychological trait such as
the experience of spiritual abuse, we chose instead to adopt language of
``severity.'' We used person parameters and scale scores to represent
participants' severity of abuse and to describe item parameters as
severity parameters.

The RSM estimates the probability of a person choosing among several
response options (i.e.~Never, Once or twice) given two values: the
severity of abuse represented by the item, and the person's severity of
the latent trait (spiritual abuse). The increase in severity involved in
choosing between two response categories is called a ``step parameter''.
The RSM makes two assumptions of the response categories. One is that
all items use the same rating scale. The other is that the same response
categories distinguish persons equally well for each item. For these
reasons the RSM estimates the same step parameters for all the items.

To convey this information visually, \textbf{?@fig-RM-plot1} presents
plots of the category response functions of two items. In both, the
horizontal axis is the scale of spiritual abuse expressed in logits
ranging from -2 to +2. The vertical axis is the probability of selecting
a given response (i.e., ``Always'') given the person's overall severity
of spiritual abuse. The blue line is the probability of selecting
Category 1 (``Never'') for persons with less severe spiritual abuse.
Persons with less severe spiritual abuse are most likely to select
``Never'' while those with more severe abuse are most likely to select
``Always.'' Between these two extremes are persons with more moderate or
average spiritual abuse. These persons are more likely to Categories 3
or 4, ``Sometimes'' or ``Often'', respectively. Both plots show curves
that display identical step parameters.

\begin{verbatim}
Iteration in WLE/MLE estimation  1   | Maximal change  2.7551 
Iteration in WLE/MLE estimation  2   | Maximal change  0.8593 
Iteration in WLE/MLE estimation  3   | Maximal change  0.4864 
Iteration in WLE/MLE estimation  4   | Maximal change  0.1926 
Iteration in WLE/MLE estimation  5   | Maximal change  0.028 
Iteration in WLE/MLE estimation  6   | Maximal change  0.0012 
Iteration in WLE/MLE estimation  7   | Maximal change  0 
----
 WLE Reliability= 0.963 
\end{verbatim}

\begin{figure}

\centering{

\includegraphics{results_files/figure-pdf/fig-RM-plot1-1.pdf}

}

\caption{\label{fig-RM-plot1-1}Category Response Plots}

\end{figure}%

\begin{figure}

\centering{

\includegraphics{results_files/figure-pdf/fig-RM-plot1-2.pdf}

}

\caption{\label{fig-RM-plot1-2}Category Response Plots}

\end{figure}%

These plots point out a step reversal between ``Never'' and ``Once or
twice''. Persons reporting the least severe spiritual abuse were most
likely to report ``Never'' to this prompt of medical care being
postponed or withheld for religious reasons. Persons farther along the
scale who have experienced ``average'' average spiritual abuse should be
more likely to report ``Once or twice'' but were still more likely to
report ``Never''. This finding suggests the ``Never'' and ``Once or
twice'' categories might be collapsed.

\subsection*{Item Severity Parameters and Fit
Statistics}\label{item-severity-parameters-and-fit-statistics}
\addcontentsline{toc}{subsection}{Item Severity Parameters and Fit
Statistics}

Table~\ref{tbl-RSM-item-tbl} reports item parameters and fit statistics
for the SHAS items. The first column, xsi, reports the severity
parameter of the item, and we have sorted the items by this value from
least to most severe to facilitate comparison between items.
Table~\ref{tbl-RSM-item-tbl} also reports fit statistics for the items.
These statistics reflect how closely the observed patterns of item
responses fit the patterns of item responses predicted by the RSM. These
fit statistics are chi-square statistics which examine the cumulative
difference the observed pattern of item responses and the pattern of
item responses that the model would expect.

Two fit statistics commonly used in IRT models are the infit mean square
and the outfit mean square (Bond \& Fox, 2015). The infit statistic
places greater emphasis on unexpected responses that are close to the
persons and item location. The outfit is sensitive to unexpected
responses that are far from the location. The expected value of infit or
outfit for each item is 1.0, with a range of acceptable values ranging
from 0.5 to 1.5. Values outside these boundaries indicate a lack of fit
between items and the model. All but one of the 60 items had infit and
outfit statistics within the acceptable range.

\begin{longtable}[t]{>{\raggedright\arraybackslash}p{11cm}ccccc}

\caption{\label{tbl-RSM-item-tbl}Estimated Item Parameters for the
Rating Scale Model and Item Chi-Square Fit Statistics}

\tabularnewline

\toprule
\multicolumn{2}{c}{ } & \multicolumn{2}{c}{Item Parameters} & \multicolumn{2}{c}{Fit Statistics} \\
\cmidrule(l{3pt}r{3pt}){3-4} \cmidrule(l{3pt}r{3pt}){5-6}
  & subscale & xsi & se.xsi & Outfit & Infit\\
\midrule
\endfirsthead
\multicolumn{6}{@{}l}{\textit{(continued)}}\\
\toprule
  & subscale & xsi & se.xsi & Outfit & Infit\\
\midrule
\endhead

\endfoot
\bottomrule
\endlastfoot
*EQ22 - Expected to follow pastor/leader rules re: dating/marriage/sex & controlling leadership & -0.42 & 0.02 & 1.20 & 1.26\\
EQ66 - Others treated as less than due to their sexual orientation &  & -0.35 & 0.02 & 1.09 & 1.05\\
EQ25 - Feeling unable to raise questions and issues &  & -0.26 & 0.02 & 0.77 & 0.76\\
EQ35 - Taught to distrust my emotions &  & -0.19 & 0.02 & 1.06 & 0.97\\
*IQ77 - Anger upon reflecting on negative religious experiences & internal distress & -0.13 & 0.02 & 0.72 & 0.69\\
\addlinespace
EQ39 - Made to feel shame over natural sexual desires (not actions) &  & -0.12 & 0.02 & 1.11 & 1.12\\
*IQ80 - Feeling isolated & internal distress & -0.11 & 0.02 & 0.94 & 0.87\\
*IQ69 - Lack of spiritual direction/purpose & internal distress & -0.05 & 0.02 & 1.35 & 1.12\\
*IQ71 - Avoiding religious activities/settings to reduce distressing feelings & internal distress & -0.04 & 0.02 & 0.92 & 0.92\\
EQ38 - Unrealistic demands placed on my moral/religious behavior &  & 0.00 & 0.02 & 0.74 & 0.77\\
\addlinespace
EQ37 - Made to feel less spiritually mature than pastor/leadership &  & 0.01 & 0.02 & 0.90 & 0.90\\
EQ54 - Being explicitly taught to distrust my intuitions &  & 0.03 & 0.02 & 0.93 & 0.93\\
IQ76 - Feeling I wasted years of my life in a church/set of beliefs &  & 0.15 & 0.02 & 1.06 & 1.09\\
EQ32 - Feeling unable to express unhappiness &  & 0.17 & 0.02 & 0.91 & 0.89\\
EQ61 - Developmentally-inappropriate/anxious end times descriptions taught to young children &  & 0.18 & 0.02 & 1.10 & 1.14\\
\addlinespace
EQ36 - Mental/physical problems interpreted as spiritual/moral weakness &  & 0.18 & 0.02 & 0.83 & 0.82\\
*IQ74 - Sadness over the loss of my faith/religious community & internal distress & 0.19 & 0.02 & 1.33 & 1.18\\
*EQ45 - Developmentally-inappropriate/anxious Hell/Satan/demons taught to young children & spiritual violence & 0.20 & 0.02 & 1.08 & 1.10\\
*IQ78 - A lack of self-worth & internal distress & 0.20 & 0.02 & 1.08 & 1.02\\
EQ18 - Church/pastor discourage critical thinking &  & 0.22 & 0.02 & 0.81 & 0.84\\
\addlinespace
EQ56 - Being made to feel I was crazy/weird for doubts/questions &  & 0.24 & 0.02 & 0.68 & 0.71\\
EQ44 - Love/acceptance offered only for high spiritual/moral performance &  & 0.36 & 0.02 & 0.75 & 0.76\\
EQ42 - Seeing others shamed/shunned by pastor/leader/group &  & 0.38 & 0.02 & 0.75 & 0.73\\
*EQ7 - Pastor speaking on God's behalf & controlling leadership & 0.45 & 0.02 & 1.36 & 1.26\\
*EQ58 - Terror/horror used to motivate religious decisions & spiritual violence & 0.45 & 0.02 & 0.91 & 0.94\\
\addlinespace
EQ11 - Disagree w/pastor portrayed as evil &  & 0.47 & 0.02 & 0.82 & 0.85\\
*IQ73 - Self-hatred or self-loathing & internal distress & 0.51 & 0.02 & 1.18 & 1.16\\
EQ29 - Feeling special when in pastor’s good graces; otherwise ignored &  & 0.52 & 0.02 & 1.10 & 1.08\\
*EQ34 - Scripture used to justify physical punishment/severe discipline & spiritual violence & 0.53 & 0.02 & 1.22 & 1.17\\
*IQ79 - Distrust of God & harmful God-image & 0.62 & 0.02 & 1.34 & 1.21\\
\addlinespace
EQ53 - Scripture used to justify abusive parent-child behavior &  & 0.64 & 0.02 & 0.96 & 1.01\\
*EQ51 - Extreme pressure to be pastor/missionary/spiritual leader & controlling leadership & 0.67 & 0.02 & 1.42 & 1.38\\
*EQ19 - Behavior excessively monitored by pastor/group & controlling leadership & 0.69 & 0.02 & 0.86 & 0.93\\
EQ63 - Shamed by pastor/group for raising questions or concerns &  & 0.72 & 0.02 & 0.76 & 0.78\\
EQ40 - Threatening Divine punishment to keep group members in line &  & 0.74 & 0.02 & 0.93 & 1.02\\
\addlinespace
*EQ28 - Leadershi/group protecting abusive individuals & maintain system & 0.76 & 0.02 & 0.92 & 0.90\\
EQ64 - Witnessing women pressured to stay in unfaithful/abusive marriages &  & 0.77 & 0.02 & 1.01 & 1.00\\
EQ50 - Treated as less than because of my gender & gender discrimination & 0.79 & 0.02 & 1.95 & 1.92\\
*IQ82 - Having trouble navigating life outside my church/community & internal distress & 0.80 & 0.02 & 1.17 & 1.16\\
EQ21 - Taught I would risk Hell if left my church/group &  & 0.83 & 0.02 & 1.30 & 1.39\\
\addlinespace
EQ59 - Pastor/group blame victim for their abuse &  & 0.87 & 0.02 & 0.76 & 0.81\\
*EQ62 - Scripture used to justify physical violence & spiritual violence & 0.88 & 0.02 & 1.08 & 1.11\\
EQ20 - Members pressured to give money despite financial hardship &  & 0.88 & 0.02 & 1.14 & 1.27\\
EQ57 - Developing mental/physical ailments from conforming to group/leader’s expectation &  & 0.89 & 0.02 & 0.94 & 1.08\\
*EQ16 - Church/community abandoning me in difficult time & maintain system & 0.92 & 0.02 & 1.02 & 1.06\\
\addlinespace
EQ49 - Leadership/group protecting and elevating abusive individuals &  & 0.92 & 0.02 & 0.89 & 0.93\\
*IQ70 - Feeling betrayed by God & harmful God-image & 0.95 & 0.02 & 1.39 & 1.27\\
*EQ52 - Blamed for harm I suffered rather than blaming who harmed me & maintain system & 0.98 & 0.02 & 0.80 & 0.89\\
IQ68 - Anxiety attacks triggered by religious stimuli &  & 1.03 & 0.02 & 0.99 & 1.06\\
EQ30 - Deterred from seeking mental health treatment/counseling/medication &  & 1.03 & 0.02 & 0.95 & 1.03\\
\addlinespace
EQ43 - Shamed by pastor/group for poor spiritual/moral performance &  & 1.03 & 0.02 & 0.81 & 0.84\\
*EQ31 - Shunned/ignored by pastor/church/group & maintain system & 1.04 & 0.02 & 1.13 & 1.09\\
EQ23 - Prayer replacing needed medical interventions &  & 1.19 & 0.02 & 1.06 & 1.08\\
*EQ24 - Expected to consult pastor/leader before making non-religious decisions & controlling leadership & 1.24 & 0.02 & 1.02 & 1.08\\
*EQ26 - Pressured to forgive abuser while abuse was ongoing & maintain system & 1.26 & 0.02 & 1.36 & 1.30\\
\addlinespace
EQ10 - Asked to give up personal goals by pastor &  & 1.32 & 0.02 & 1.11 & 1.19\\
EQ65 - Cut off/shunned by more religious family members &  & 1.33 & 0.02 & 1.28 & 1.26\\
IQ81 - Nightmares about my negative religious experiences &  & 1.42 & 0.02 & 0.90 & 0.97\\
*IQ72 - Feeling as if God harmed me directly & harmful God-image & 1.64 & 0.02 & 1.13 & 1.22\\
EQ47 - Medical care being postponed/withheld for religious reasons &  & 1.94 & 0.03 & 1.10 & 1.16\\*

\end{longtable}

\begin{figure}

\centering{

\includegraphics{results_files/figure-pdf/fig-RSM-item-histogram-1.pdf}

}

\caption{\label{fig-RSM-item-histogram}Distribution of Item Severity
Parameters}

\end{figure}%

\subsection*{Reliability}\label{reliability}
\addcontentsline{toc}{subsection}{Reliability}

The RSM expresses internal consistency reliability as Rasch person
separation. Based on Table X, the SHAS has a high Rasch person
separation reliability value of .963, indicating that the estimated RSM
scale discriminated well between persons with varying severity of
spiritual abuse.

The SHAS also has a high Rasch item separation reliability with Rel =
.XX, χ2 (XX) = XXXX.X, p \textless{} .XX, implying that the items have a
good spread in terms of item ordering and hierarchy.

\subsection*{Item-Person Map}\label{item-person-map}
\addcontentsline{toc}{subsection}{Item-Person Map}

Figure~\ref{fig-RSM-WrightMap} displays an item-person map (also called
a ``Wright map'') that shows the item severities of the SHAS items and
the person severities for each person who completed the survey. All
these severities are estimated in logits (log odds units) as the unit of
measures. This means the higher the logit value for an item is, the less
likely it was for a person to endorse that particular item. On the other
hand, higher logit values for each person indicate more severe spiritual
abuse. On the variable map, the mean item severity is constrained to be
0.00 with person severities being relative to that mean item severity
While column 1 shows the latent continuum in terms of logit values as
the unit of measurement underlying the SHAS, columns 2 and 3 represent
the severities of the people and the items.

Figure~\ref{fig-RSM-WrightMap} shows good overlap between the person
trait and item severity as evidenced by the match between the mean of
the person severity (M = − 0.58) and the mean of the item severity (M =
0.00). This suggests that the majority of the items were appropriate for
the sample. While person severity measures range from X.XX logits to −
X.XX logits (M = − 0.XX, SD = 0.XX, N = XXXX), the item severities range
from X.XX logits to − X.XX logits (M = X.XX, SD = 0.XX, N = XXXX). The
item representing the most severe spiritual abuse for participants is
Item X, ``I feel \ldots{} XXX'' with the severity at X.XX logits (also
reported in Appendix Table 2). By contrast the item representing the
least severe spiritual abuse is item XX, ``I would \ldots{}'', with −
1.08 logits.

\begin{figure}

\centering{

\includegraphics{results_files/figure-pdf/fig-RSM-WrightMap-1.pdf}

}

\caption{\label{fig-RSM-WrightMap}Item-Person Map}

\end{figure}%

\subsection*{Model-Data Fit}\label{model-data-fit}
\addcontentsline{toc}{subsection}{Model-Data Fit}

To assess the overall fit of the RSM to the SHAS item response data, we
use two statistics. The first is the Root Mean Square Error of
Approximation (RMSEA). The suggested cutoff for RMSEA is .06. The second
is the CFI. The suggested cutoff for the CFI is .95. These fit
statistics are presented in Table RSM Model Fit.

\begin{verbatim}
'log Lik.' -225573.6 (df=64)
\end{verbatim}

The obtained RMSEA value of .09 (95\% CI{[}.092, .093{]}) exceeds the
cutoff of .06. The CFI of .953 meets the recommended .95 threshold.
Together these statistics evidence that the RSM fits the SHAS items.

\section*{DIF Analysis}\label{dif-analysis}
\addcontentsline{toc}{section}{DIF Analysis}

\markright{DIF Analysis}

Differential item functioning (DIF) examines how an item functions
differently for people of equal standing on the trait. In this case, we
conducted DIF analysis of the SHAS items to examine item bias by gender,
and age, and race. {[}We applied logistic ordinal regression with IRT
scoring. We used the Chi-squared likelihood-ratio statistic as the
initial DIF detection criteria (alpha \textless{} 0.01) and a cut-off of
McFadden pseudo R2Δ ≥ 0.02 in model comparisons to determine substantial
DIF, a reasonable threshold used in the development of self-reported
health outcomes.{]}

In this analysis, we examined the degree to which there was evidence of
uniform differential item functioning (DIF) between two subgroups of
participants (group 1 and group 2). As a first step we examined item
severity estimates and standard errors specific to each subgroup using
the Rating Scale Model Andrich (1978). We estimated item severity for
the two subgroups using a combined analysis of both subgroups and then
estimating the group-specific item severities and standard errors are
shown in Table X.

To examine the differences in item severity between subgroups, we
calculated standardized differences following Wright and Masters (1982)
as follows:

{[}equation{]}

where\\
z is the standardized difference, d1 is the item severity specific to
Subgroup 1, d2 is the item severity specific to Subgroup 2,\\
s e 2 1 is the standard error of the item severity specific to Subgroup
1, and\\
s e 2 2 is the standard error of the item severity specific to Subgroup
2. Using the formulation of the z statistic, higher values of z indicate
higher item locations (more-severe to endorse) for Subgroup 1 compared
to Subgroup 2.

Figure X shows a plot of the z-statistics for the four items; these
values are also presented numerically in Table X. In the figure, the
x-axis shows the item identification numbers, and the y-axis shows the
value of the z-statistic. Boundaries at +2 and -2 are indicated using
dashed horizontal lines to highlight statistically significant
differences in item severity between subgroups. Examination of these
results indicates that the items were not significantly different in
severity between the two subgroups. In addition, there were both
positive and negative z statistics, indicating that although the
differences in item severity were not significant, there were some items
that were easier to endorse for Subgroup 1 and others that were easier
to endorse for Subgroup 2.

Figure 1: Plot of Standardized Differences for Items between Subgroups

To further explore the differences in item severity between the two
subgroups, Figure 2 shows a scatterplot of the item locations between
the two subgroups. In the plot, the item severity for Subgroup 1 is
shown on the x-axis, and the item severity for Subgroup 2 is shown on
the y-axis. Individual items are indicated using open circle plotting
symbols. A solid identity line is included to highlight deviations from
invariant item severities between the two groups: Points that fall below
this line indicate that items were easier to endorse (lower item
measures) for Subgroup 2, and points that fall above the line indicate
that items were easier to endorse (lower item measures) for Subgroup 1.
Dashed lines are also included to indicate a 95\% confidence interval
for the difference between the item measures, following Luppescu (1995).

Figure 2: Scatterplot of Subgroup-Specific Item Severities

Finally, Figure 3 is a bar plot that illustrates the direction and
magnitude of the differences in item severities between subgroups. In
the plot, each bar represents the difference in severity between
subgroups for an individual item, ordered by the item sequence in the
survey. Bars that point to the left of the plot indicate that the item
was easier to endorse for Subgroup 1, and bars that point to the right
of the plot indicate that the item was easier to endorse for Subgroup 2.
Dashed vertical lines are plotted that show values of +0.5 and -0.5
logits as an indicator of substantial differences in item severity
between subgroups.

Figure 3: Bar Plot of Differences In Item Severity Between Subgroups

\bookmarksetup{startatroot}

\chapter*{Discussion}\label{discussion}
\addcontentsline{toc}{chapter}{Discussion}

\markboth{Discussion}{Discussion}

The purpose of this study was to further validate and enhance the
Spiritual Harm and Abuse Scale (SHAS) with the use of the Rasch Rating
Scale model. The present study improves upon the sample dependency of
factor analysis results and contributes new understanding about the
psychometric characteristics of the SHAS.

In this study, we examined unidimensionality, Rasch person/item
separation reliability, model-data fit, distribution of the variable map
based on the locations of persons and items, and individual item
qualities of the SHAS. In terms of dimensionality, we found satisfactory
evidence to unidimensionality to apply the Rasch Rating Scale model. We
found the SHAS highly reliable using Rasch person/item separation
reliability statistics, indicating that most of the items are sensitive
enough to capture different levels of spiritual abuse and most persons'
observed values overlap with the expected values. Similarly, there was
close correspondence between the data and the Rasch Rating Scale model
except for the outliers in the XXX ends of the continuum. This indicates
reasonable model-data fit. Regarding the variable map, our results
indicate that the SHAS, overall, captures a range of item severities
that reflect the order structure of spiritual abuse. However, we also
found a shortage of items at the very high and low ends of the
continuum. The SHAS can capture people who have experienced moderate
severity of spiritual abuse, but it is not sensitive enough to
discriminate among people who have experienced very severe and less
severe spiritual abuse. \textbackslash{[}For example, based on our Rasch
Rating scale model analysis, someone with a measure of − 3.00 logits and
another with a measure of − 2.00 logits would be more likely to respond
to the SHAS in the same way because there is no item on the scale that
targets very low levels of spiritual abuse. Hence, we would not be able
to detect which person has relatively more or less spiritual abuse in
comparison to each other. It would be similarly impossible to
distinguish levels of spiritual abuse between a person with a measure of
3.00 logits and another with a measure of 2.00 logits, and decide who
has experienced more severe spiritual abuse due to lack of items at the
high end of the continuum.\textbackslash{]} Finally, in terms of item
quality, XX of the XX items exhibited good psychometric quality, whereas
XX items were misfits. XXX items were redundant because they provided
the exact same information with identical pairs of each item, displaying
the same contributions to the sensitivity of the SHAS. Hence, they can
be removed from the scale to shorten the duration of administration.

Use of the Rasch measurement model in our study provided a powerful
analysis of the functioning of the SHAS at the item level and offered
in-depth insights on applicability of the scale in wider contexts due to
its sample independent nature. Whereas factor analytic approaches such
as exploratory factor analysis and confirmatory factor analysis do not
provide invariant measurement due to their sample dependency, the Rasch
measurement model is independent from the items that form the scale and
independent from the sample that the scale is administered to. This
feature will increase the applicability of the scale in wider contexts
and replicability of the present study in different populations due to
the possibility of direct application of the standards developed in one
study to data administered in another study (Yan and Mok 2012).

In conclusion, the results of this study provide strong evidence for the
validation of the SHAS despite the need for considering the deletion of
XX items that are misfits and of XX items that are redundant. The
revision or development of new items can also be considered to capture
gradations of challenges at the very high and low ends of the continuum
to detect people with very high and low levels of spiritual abuse.

\section*{Implications and future
directions}\label{implications-and-future-directions}
\addcontentsline{toc}{section}{Implications and future directions}

\markright{Implications and future directions}

This study presents the validation of the SHAS using the Rasch
measurement model, for assessing spiritual abuse in adults. The findings
have important implications for clinicians and researchers who are
concerned about spiritual abuse. The finding that the majority of the
SHAS items were of high quality implies that the items provide adequate
information to measure moderate levels of spiritual abuse. However, we
found the SHAS less able to distinguish people with very high and low
levels of spiritual abuse. Hence, for now clinicians and researchers
should be cautious when interpreting their results for people who have
very high and low spiritual abuse levels.

We echo the same limitations of these data noted in Koch \& Edstrom
(2022). The SHAS was developed on a largely white and Protestant/former
Protestant population. Possibly major types or sub-themes of spiritual
abuse not captured here are more prominent in other religious
traditions. This would be a fruitful area of future study. Research into
spiritual abuse among the LGBTQ+ community might also yield insights
specific to that population. Future research could also distinguish
normed scores of this scale for different populations and denominations.

The final screener has the potential to both improve therapy for
innumerable clients who have experienced negative religious experiences
and to contribute a knowledge base for any future clinical training
around working with spiritually abused clients. The scale may also
contribute to future research that could be critical in understanding
not only the mechanisms and experience of spiritual abuse but also
perhaps the enduring power of religion, both to heal and to harm.

\bookmarksetup{startatroot}

\chapter*{Data Availability}\label{data-availability}
\addcontentsline{toc}{chapter}{Data Availability}

\markboth{Data Availability}{Data Availability}

Data will be made available in accordance with xxx. Within one year of
publication of manuscripts addressing the aims of the grant,
investigators will make a deidentified, anonymized dataset available to
the public.

\bookmarksetup{startatroot}

\chapter*{Code Availability}\label{code-availability}
\addcontentsline{toc}{chapter}{Code Availability}

\markboth{Code Availability}{Code Availability}

Code for the analyses presented in this paper are available at
https://github.com/jackbhuber/spiritual\_abuse.

\bookmarksetup{startatroot}

\chapter*{Acknowledgments}\label{acknowledgments}
\addcontentsline{toc}{chapter}{Acknowledgments}

\markboth{Acknowledgments}{Acknowledgments}

This manuscript was not supported by grant funding. The views expressed
in this article are those of the authors and do not necessarily reflect
the position or policy of the Swedish Medical Center or Northwest
University.

\bookmarksetup{startatroot}

\chapter*{Funding}\label{funding}
\addcontentsline{toc}{chapter}{Funding}

\markboth{Funding}{Funding}

This research was not supported by grant funding.

\bookmarksetup{startatroot}

\chapter*{Author Information}\label{author-information}
\addcontentsline{toc}{chapter}{Author Information}

\markboth{Author Information}{Author Information}

\section*{Affiliations}\label{affiliations}
\addcontentsline{toc}{section}{Affiliations}

\markright{Affiliations}

Jack Huber Department of Pharmacy, Swedish Medical Center, Seattle, USA

Dan Koch Department of Psychology, Northwest University, Kirkland, USA

\section*{Corresponding Author}\label{corresponding-author}
\addcontentsline{toc}{section}{Corresponding Author}

\markright{Corresponding Author}

Please direct correspondence to Jack Huber at
\href{mailto:jack.bernard.huber@gmail.com}{\nolinkurl{jack.bernard.huber@gmail.com}}.

\bookmarksetup{startatroot}

\chapter*{Ethics Declarations}\label{ethics-declarations}
\addcontentsline{toc}{chapter}{Ethics Declarations}

\markboth{Ethics Declarations}{Ethics Declarations}

\section*{Conflict of Interest}\label{conflict-of-interest}
\addcontentsline{toc}{section}{Conflict of Interest}

\markright{Conflict of Interest}

The authors declare no conflict of interest.

\bookmarksetup{startatroot}

\chapter*{References}\label{references}
\addcontentsline{toc}{chapter}{References}

\markboth{References}{References}

\phantomsection\label{refs}
\begin{CSLReferences}{1}{0}
\bibitem[\citeproctext]{ref-Andrich1978}
Andrich, D. (1978). A rating formulation for ordered response
categories. \emph{Psychometrika}, \emph{43}, 561--573.

\bibitem[\citeproctext]{ref-BondFox2015}
Bond, T. G., \& Fox, C. M. (2015). \emph{Applying the rasch model:
Fundamental measurement in the human sciences}. Routledge.

\bibitem[\citeproctext]{ref-Briere-1989}
Briere, J., \& Runtz, M. (1989). The trauma symptom checklist (TSC-33):
Early data on a new scale. \emph{Journal of Interpersonal Violence},
\emph{4}(2), 151--163. \url{https://doi.org/10.1177/088626089004002002}

\bibitem[\citeproctext]{ref-Carlson-2011}
Carlson, E. B., Smith, S. R., Palmieri, P. A., Dalenberg, C. J., Ruzek,
J. I., Kimerling, R., Burling, T. A., \& Spain, D. A. (2011).
Development and validation of a brief self-report measure of trauma
exposure: The trauma history screen. \emph{Psychological Assessment},
\emph{23}, 463--477. \url{https://doi:10.1037/a0022294}

\bibitem[\citeproctext]{ref-R-mirt}
Chalmers, P. (2024). \emph{Mirt: Multidimensional item response theory}.
\url{https://github.com/philchalmers/mirt}

\bibitem[\citeproctext]{ref-Doyle-2006}
Doyle, T. P. (2006). Clericalism: Enabler of clergy sexual abuse.
\emph{Pastoral Psychology}, \emph{54}, 189--213.
\url{https://doi:10.1007/s11089-006-6323-x}

\bibitem[\citeproctext]{ref-Yen1984}
Effects of local item dependence on the fit and equating performance of
the three-parameter logistic model. (1984). \emph{Applied Psychological
Measurement}, \emph{8}(2), 125--145.

\bibitem[\citeproctext]{ref-EmbretsonReise2000}
Embretson, S. E., \& Reise, S. P. (2000). \emph{Item response theory for
psychologists}. Lawrence Erlbaum Associates.

\bibitem[\citeproctext]{ref-Foster-2015}
Foster, K. A., Bowland, S. E., \& Vosler, A. N. (2015). All the pain
along with all the joy: Spiritual resilience in lesbian and gay
christians. \emph{American Journal of Community Psychology}, \emph{55},
191--201. \url{https://doi:10.1007/s10464-015-9704-4}

\bibitem[\citeproctext]{ref-Gubi-2009}
Gubi, P. M., \& Jacobs, R. (2009). Exploring the impact on counsellors
of working with spiritually abused clients. \emph{Mental Health,
Religion \& Culture}, \emph{12}(2), 191--204.
\url{https://doi:10.1080/13674670802441509}

\bibitem[\citeproctext]{ref-Keller2016}
Keller, K. H. (2016). \emph{Development of a spiritual abuse
questionnaire} {[}PhD thesis{]}. Texas Woman's University.

\bibitem[\citeproctext]{ref-Kira-2008}
Kira, I. A., Lewandowski, L., Templin, T., Ramaswamy, V., Ozkan, B., \&
Mohanesh, J. (2008). Measuring cumulative trauma dose, types, and
profiles using a development-based taxonomy of traumas.
\emph{Traumatology}, \emph{14}(2), 62--87.
\url{https://doi.org/10.1177/1534765608319324}

\bibitem[\citeproctext]{ref-KochEdstrom2022}
Koch, D., \& Edstrom, L. (2022). Development of the spiritual harm and
abuse scale. \emph{Journal for the Scientific Study of Religion}.

\bibitem[\citeproctext]{ref-Kvarfordt-2010}
Kvarfordt, C. L. (2010). Spiritual abuse and neglect of youth:
Reconceptualizing what is known through an investigation of
practitioners' experiences. \emph{Journal of Religion \& Spirituality in
Social Work: Social Thought}, \emph{29}(2), 143--164.
\url{https://doi:10.1080/15426431003708287}

\bibitem[\citeproctext]{ref-R-eRm}
Mair, P., Rusch, T., Hatzinger, R., \& Maier, M. J. (2024). \emph{eRm:
Extended rasch modeling}. \url{https://CRAN.R-project.org/package=eRm}

\bibitem[\citeproctext]{ref-Nobakht-2017}
Nobakht, H. N., \& Dale, K. Y. (2017). The importance of
religious/ritual abuse as a traumatic predictor of dissociation.
\emph{Journal of Interpersonal Violence}, \emph{33}(23), 3575--3588.
\url{https://doi:10.1177/0886260517723747}

\bibitem[\citeproctext]{ref-Oakley-2014}
Oakley, L. R., \& Kinmond, K. S. (2014). Developing safeguarding policy
and practice for spiritual abuse. \emph{The Journal of Adult
Protection}, \emph{16}(2), 87--95.
\url{https://doi:10.1108/jap-07-2013-0033}

\bibitem[\citeproctext]{ref-Oakley-2018}
Oakley, Lv., Kinmond, K., \& Humphreys, J. (2018). Spiritual abuse in
christian faith settings: Definition, policy and practice guidance.
\emph{The Journal of Adult Protection}, \emph{20}(3/4), 144--154.
\url{https://doi:10.1108/jap-03-2018-0005}

\bibitem[\citeproctext]{ref-Pearl-1994}
Pearl, P. (1994). Emotional abuse. In J. A. Monteleone \& A. E. Brodeur
(Eds.), \emph{Child maltreatment: A clinical guide and reference}. G.W.
Medical Publisher.

\bibitem[\citeproctext]{ref-R-base}
R Core Team. (2024). \emph{R: A language and environment for statistical
computing}. R Foundation for Statistical Computing.
\url{https://www.R-project.org/}

\bibitem[\citeproctext]{ref-Rasch1960}
Rasch, G. (1960). \emph{Probabilistic models for some intelligence and
attainment tests}. University of Chicago Press.

\bibitem[\citeproctext]{ref-Reckase1979}
Reckase, M. D. (1979). Unifactor latent trait models applied to
multifactor tests: Results and implications. \emph{Journal of
Educational Statistics}, \emph{4}, 207--230.

\bibitem[\citeproctext]{ref-ReiseBonifayHaviland2013}
Reise, S. P., Bonifay, W. E., \& Haviland, M. G. (2013). Scoring and
modeling psychological measures in the presence of multidimensionality.
\emph{Journal of Personality Assessment}, \emph{95}(2), 129--140.
\url{https://doi.org/10.1080/00223891.2012.725437}

\bibitem[\citeproctext]{ref-ReiseMooreHaviland2013}
Reise, S. P., Moore, T. M., \& Haviland, M. G. (2013). Applying
unidimensional item response theory models to psychological data. In
\emph{APA handbook of testing and assessment in psychology: Vol. 1. Test
theory and testing and assessment in industrial and organizational
psychology} (pp. 101--119).

\bibitem[\citeproctext]{ref-ReiseWaller2009}
Reise, S. P., \& Waller, N. G. (2009). Item response theory and clinical
measurement. \emph{Annual Review of Clinical Psychology}, \emph{5},
27--48. \url{https://doi.org/10.1146/annurev.clinpsy.032408.153553}

\bibitem[\citeproctext]{ref-R-psych}
Revelle, W. (2024). \emph{Psych: Procedures for psychological,
psychometric, and personality research}.
\url{https://personality-project.org/r/psych/}

\bibitem[\citeproctext]{ref-R-TAM}
Robitzsch, A., Kiefer, T., \& Wu, M. (2024). \emph{TAM: Test analysis
modules}. \url{http://www.edmeasurementsurveys.com/TAM/Tutorials/}

\bibitem[\citeproctext]{ref-Rodruxedguez-Carballeira-2015}
Rodríguez-Carballeira, A., Saldaña, O., Almendros, C., Martín-Peña, J.,
Escartín, J., \& Porrúa-García, C. (2015). Group psychological abuse:
Taxonomy and severity of its components. \emph{The European Journal of
Psychology Applied to Legal Context}, \emph{7}(1), 31--39.
\url{https://doi:10.1016/j.ejpal.2014.11.001}

\bibitem[\citeproctext]{ref-Sanders-1995}
Sanders, B., \& Becker-Lausen, E. (1995). The measurement of
psychological maltreatment: Early data on the child abuse and trauma
scale. \emph{Child Abuse \& Neglect}, \emph{19}(3), 315--323.
\url{https://doi.org/10.1016/S0145-2134(94)00131-6}

\bibitem[\citeproctext]{ref-Stevens-2019}
Stevens, J. M., Arzoumanian, M. A., Greenbaum, B., Schwab, B. M., \&
Dalenberg, C. J. (2019). Relationship of abuse by religious authorities
to depression, religiosity, and child physical abuse history in a
college sample. \emph{Psychological Trauma: Theory, Research, Practice,
and Policy}, \emph{11}(3), 292--299.
\href{https://doi:\%2010.1037/tra0000421}{https://doi:
10.1037/tra0000421}

\bibitem[\citeproctext]{ref-Swindle-2017}
Swindle, P. J. (2017). \emph{A twisting of the sacred: The lived
experience of religious abuse} {[}PhD thesis, University of North
Carolina at Greensboro{]}.
\url{https://libres.uncg.edu/ir/uncg/f/Swindle_uncg_0154D_12186.pdf}

\bibitem[\citeproctext]{ref-Ward-2011}
Ward, D. J. (2011). The lived experience of spiritual abuse.
\emph{Mental Health, Religion \& Culture}, \emph{14}(9), 899--915.
\url{https://doi:10.1080/13674676.2010.536206}

\bibitem[\citeproctext]{ref-Winell-nd}
Winell, M. (n.d.). \emph{Understanding religious trauma syndrome: Trauma
from religion}. Retrieved at
\url{https://legacy.babcp.com/Review/RTS-Trauma-from-Religion.aspx}
(2020/11/29).

\end{CSLReferences}




\end{document}
