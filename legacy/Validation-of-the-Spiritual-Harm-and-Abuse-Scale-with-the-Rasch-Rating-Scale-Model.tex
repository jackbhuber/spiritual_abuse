% Options for packages loaded elsewhere
\PassOptionsToPackage{unicode}{hyperref}
\PassOptionsToPackage{hyphens}{url}
\PassOptionsToPackage{dvipsnames,svgnames,x11names}{xcolor}
%
\documentclass[
  letterpaper,
  DIV=11,
  numbers=noendperiod]{scrreport}

\usepackage{amsmath,amssymb}
\usepackage{lmodern}
\usepackage{iftex}
\ifPDFTeX
  \usepackage[T1]{fontenc}
  \usepackage[utf8]{inputenc}
  \usepackage{textcomp} % provide euro and other symbols
\else % if luatex or xetex
  \usepackage{unicode-math}
  \defaultfontfeatures{Scale=MatchLowercase}
  \defaultfontfeatures[\rmfamily]{Ligatures=TeX,Scale=1}
\fi
% Use upquote if available, for straight quotes in verbatim environments
\IfFileExists{upquote.sty}{\usepackage{upquote}}{}
\IfFileExists{microtype.sty}{% use microtype if available
  \usepackage[]{microtype}
  \UseMicrotypeSet[protrusion]{basicmath} % disable protrusion for tt fonts
}{}
\makeatletter
\@ifundefined{KOMAClassName}{% if non-KOMA class
  \IfFileExists{parskip.sty}{%
    \usepackage{parskip}
  }{% else
    \setlength{\parindent}{0pt}
    \setlength{\parskip}{6pt plus 2pt minus 1pt}}
}{% if KOMA class
  \KOMAoptions{parskip=half}}
\makeatother
\usepackage{xcolor}
\setlength{\emergencystretch}{3em} % prevent overfull lines
\setcounter{secnumdepth}{5}
% Make \paragraph and \subparagraph free-standing
\ifx\paragraph\undefined\else
  \let\oldparagraph\paragraph
  \renewcommand{\paragraph}[1]{\oldparagraph{#1}\mbox{}}
\fi
\ifx\subparagraph\undefined\else
  \let\oldsubparagraph\subparagraph
  \renewcommand{\subparagraph}[1]{\oldsubparagraph{#1}\mbox{}}
\fi


\providecommand{\tightlist}{%
  \setlength{\itemsep}{0pt}\setlength{\parskip}{0pt}}\usepackage{longtable,booktabs,array}
\usepackage{calc} % for calculating minipage widths
% Correct order of tables after \paragraph or \subparagraph
\usepackage{etoolbox}
\makeatletter
\patchcmd\longtable{\par}{\if@noskipsec\mbox{}\fi\par}{}{}
\makeatother
% Allow footnotes in longtable head/foot
\IfFileExists{footnotehyper.sty}{\usepackage{footnotehyper}}{\usepackage{footnote}}
\makesavenoteenv{longtable}
\usepackage{graphicx}
\makeatletter
\def\maxwidth{\ifdim\Gin@nat@width>\linewidth\linewidth\else\Gin@nat@width\fi}
\def\maxheight{\ifdim\Gin@nat@height>\textheight\textheight\else\Gin@nat@height\fi}
\makeatother
% Scale images if necessary, so that they will not overflow the page
% margins by default, and it is still possible to overwrite the defaults
% using explicit options in \includegraphics[width, height, ...]{}
\setkeys{Gin}{width=\maxwidth,height=\maxheight,keepaspectratio}
% Set default figure placement to htbp
\makeatletter
\def\fps@figure{htbp}
\makeatother
\newlength{\cslhangindent}
\setlength{\cslhangindent}{1.5em}
\newlength{\csllabelwidth}
\setlength{\csllabelwidth}{3em}
\newlength{\cslentryspacingunit} % times entry-spacing
\setlength{\cslentryspacingunit}{\parskip}
\newenvironment{CSLReferences}[2] % #1 hanging-ident, #2 entry spacing
 {% don't indent paragraphs
  \setlength{\parindent}{0pt}
  % turn on hanging indent if param 1 is 1
  \ifodd #1
  \let\oldpar\par
  \def\par{\hangindent=\cslhangindent\oldpar}
  \fi
  % set entry spacing
  \setlength{\parskip}{#2\cslentryspacingunit}
 }%
 {}
\usepackage{calc}
\newcommand{\CSLBlock}[1]{#1\hfill\break}
\newcommand{\CSLLeftMargin}[1]{\parbox[t]{\csllabelwidth}{#1}}
\newcommand{\CSLRightInline}[1]{\parbox[t]{\linewidth - \csllabelwidth}{#1}\break}
\newcommand{\CSLIndent}[1]{\hspace{\cslhangindent}#1}

\usepackage{booktabs}
\usepackage{longtable}
\usepackage{array}
\usepackage{multirow}
\usepackage{wrapfig}
\usepackage{float}
\usepackage{colortbl}
\usepackage{pdflscape}
\usepackage{tabu}
\usepackage{threeparttable}
\usepackage{threeparttablex}
\usepackage[normalem]{ulem}
\usepackage{makecell}
\usepackage{xcolor}
\KOMAoption{captions}{tableheading}
\makeatletter
\makeatother
\makeatletter
\@ifpackageloaded{bookmark}{}{\usepackage{bookmark}}
\makeatother
\makeatletter
\@ifpackageloaded{caption}{}{\usepackage{caption}}
\AtBeginDocument{%
\ifdefined\contentsname
  \renewcommand*\contentsname{Table of contents}
\else
  \newcommand\contentsname{Table of contents}
\fi
\ifdefined\listfigurename
  \renewcommand*\listfigurename{List of Figures}
\else
  \newcommand\listfigurename{List of Figures}
\fi
\ifdefined\listtablename
  \renewcommand*\listtablename{List of Tables}
\else
  \newcommand\listtablename{List of Tables}
\fi
\ifdefined\figurename
  \renewcommand*\figurename{Figure}
\else
  \newcommand\figurename{Figure}
\fi
\ifdefined\tablename
  \renewcommand*\tablename{Table}
\else
  \newcommand\tablename{Table}
\fi
}
\@ifpackageloaded{float}{}{\usepackage{float}}
\floatstyle{ruled}
\@ifundefined{c@chapter}{\newfloat{codelisting}{h}{lop}}{\newfloat{codelisting}{h}{lop}[chapter]}
\floatname{codelisting}{Listing}
\newcommand*\listoflistings{\listof{codelisting}{List of Listings}}
\makeatother
\makeatletter
\@ifpackageloaded{caption}{}{\usepackage{caption}}
\@ifpackageloaded{subcaption}{}{\usepackage{subcaption}}
\makeatother
\makeatletter
\@ifpackageloaded{tcolorbox}{}{\usepackage[many]{tcolorbox}}
\makeatother
\makeatletter
\@ifundefined{shadecolor}{\definecolor{shadecolor}{rgb}{.97, .97, .97}}
\makeatother
\makeatletter
\makeatother
\ifLuaTeX
  \usepackage{selnolig}  % disable illegal ligatures
\fi
\IfFileExists{bookmark.sty}{\usepackage{bookmark}}{\usepackage{hyperref}}
\IfFileExists{xurl.sty}{\usepackage{xurl}}{} % add URL line breaks if available
\urlstyle{same} % disable monospaced font for URLs
\hypersetup{
  pdftitle={Validation of the Spiritual Harm and Abuse Scale with the Rasch Rating Scale Model},
  pdfauthor={Jack Huber, Swedish Medical Center; Dan Koch, Northwest University},
  colorlinks=true,
  linkcolor={blue},
  filecolor={Maroon},
  citecolor={Blue},
  urlcolor={Blue},
  pdfcreator={LaTeX via pandoc}}

\title{Validation of the Spiritual Harm and Abuse Scale with the Rasch
Rating Scale Model}
\author{Jack Huber, Swedish Medical Center \and Dan Koch, Northwest
University}
\date{8/13/2022}

\begin{document}
\maketitle
\ifdefined\Shaded\renewenvironment{Shaded}{\begin{tcolorbox}[interior hidden, boxrule=0pt, frame hidden, sharp corners, breakable, enhanced, borderline west={3pt}{0pt}{shadecolor}]}{\end{tcolorbox}}\fi

\renewcommand*\contentsname{Table of contents}
{
\hypersetup{linkcolor=}
\setcounter{tocdepth}{2}
\tableofcontents
}
\bookmarksetup{startatroot}

\hypertarget{abstract}{%
\chapter*{Abstract}\label{abstract}}
\addcontentsline{toc}{chapter}{Abstract}

Spiritual harm and abuse has been a serious problem that has begun to
gain the attention of the clinical and psychological research community.
Some research has attempted to define spiritual harm and abuse and
documented lasting emotional and psychological damage to victims of
spiritual abuse. A valid scale of spiritual and religious abuse is
needed, and the nascent Spiritual Harm and Abuse Scale (SHAS) has
potential to meet this need. Koch and Edstrom (2022) developed an item
pool and conducted an initial validation study of the SHAS which
established the multidimensional structure of the item pool. This study
builds on Koch and Edstrom (2022) to further validate the SHAS by using
the Rasch Rating Scale Model.

\bookmarksetup{startatroot}

\hypertarget{introduction}{%
\chapter*{Introduction}\label{introduction}}
\addcontentsline{toc}{chapter}{Introduction}

Spiritual and religious abuse has been a serious problem which is
gaining the attention of the psychological research community. Koch and
Edstrom (2022) define spiritual abuse as ``a type of emotional and
psychological abuse perpetrated by a religious leader or group and/or
with a religious or spiritual component, usually involving coercion or
control.'' Such abuse can range from psychologically controlling
behaviors like quoting scripture or invoking divine rationale for
mistreatment (Oakley et al., 2018); to diverse emotionally abusive
behaviors such as rejecting, isolating, terrorizing, ignoring,
corrupting, verbally assaulting, and over-pressuring (Pearl, 1994); to
sexual or other physical forms of abuse by a clergy member (Doyle,
2006). Spiritual abuse can also range in specificity from abusive acts
by religious leaders to abuse perpetrated by a religious group or person
representing a religious group with a religious or spiritual component
(Swindle, 2017). Spiritual abuse may also include theologically
sanctioned mistreatment of a group of people such LGBTQ+ persons
(Foster, Bowland, \& Vosler, 2015).

Spiritual and religious abuse has serious negative effects (Ward, 2011;
Swindle, 2017; Oakley \& Kinmond, 2014; Stevens et al.~2019). Religious
leaders may be on a par with parents in their potential for abusive
harm. Further, trauma that takes place in a religious or spiritual
setting may deprive victims of religion and spirituality as a coping
strategy. There are also concerns related to secondary trauma for
clinicians working with R/S abused clients (Gubi \& Jacobs, 2009).

In light of these concerns, a valid scale of spiritual harm and abuse is
needed. Practitioners who treat clients who may be victims of religious
and spiritual abuse need an instrument for clinical use. Clinicians
could use such a tool as part of the intake process, to evaluate their
client's score in light of established norms. In addition, researchers
need a reliable measurement of spiritual abuse to examine relationships
between spiritual abuse and any number of other psychological, health,
or other measures.

Drawing from a recent unpublished scale of religious and abuse (Keller
2016) and other scales of abuse (Kira, et al.~2008; Sanders \&
Becker-Lausen, 1995; Carlson, et al.~2011; Briere \& Runtz, 1989), Koch
and Edstrom (2022) developed the Spiritual Harm \& Abuse Scale (SHAS).
These authors drew from the literature and interviews to find a wide
range of experiences and emotional states that could be described as
spiritual harm and abuse. From this work they developed a pool of 66
prompts for an online community survey which they fielded in 2021. Their
study of the factor structure revealed a handful of factors which they
used to distill items to a 27-item scale with six subscales.

In this paper, we build upon the work of Koch and Edstrom (2022) to
further validate the SHAS by using item response theory. Item response
theory can enhance clinical assessments in several ways. First, IRT
provides interval-level measures which are more appropriate for
statistical analysis than ordinal-level sum scores. Second, IRT improves
upon sum scores by providing a scale along which both items and persons
can be located. Third, measurements derived from IRT are sample-free,
meaning it is possible to estimate a person's level of the latent
construct free of the distribution of individual items and to estimate
an item's difficulty level free from the distribution of people used in
the sample. Finally, IRT improves upon single overall reliability
coefficients by providing examination of precision across the score
continuum.

The goal of this study was thus to use the Rasch Rating Scale Model (an
IRT model) to enhance the SHAS in several ways. First, we aimed to
express the SHAS as a single interval-level outcome measure of severity
of spiritual harm and abuse. Second, we aimed to enrich the diagnostic
value of the items and scale by examining severity. Third, we aimed to
examine the distribution of item severities to see where the scale
provides the most precise, reliable measurement of spiritual abuse. Such
a scale also lays the foundation to further develop the SHAS by
identifying anchor items that can calibrate with pilot items on future
forms. Finally, we aimed to establish a SHAS score that is free of
differential item functioning based on race, gender, and other
demographic variables.

\bookmarksetup{startatroot}

\hypertarget{methods}{%
\chapter*{Methods}\label{methods}}
\addcontentsline{toc}{chapter}{Methods}

\hypertarget{item-pool-development}{%
\section*{Item Pool Development}\label{item-pool-development}}
\addcontentsline{toc}{section}{Item Pool Development}

Koch and Edstrom (2022) developed an initial pool of items to measure
spiritual harm and abuse. The authors validated the construct of
spiritual harm and abuse through literature review (Kvarfordt, 2010;
Nobakht \& Dale, 2018; Oakley \& Kinmond, 2014; Rodríguez-Carballeira,
et al.~2015; Swindle, 2017; Ward, 2011, Winell, n.d.) and interviews
with survivors of religious and spiritual abuse. These interviews
revealed an array of specific examples of potential abuse not described
in the published literature such as abuse inflicted by narcissistic
persons, financial coercion, developmentally inappropriate children's
teachings, pressure to stay in physically abusive marriages, neglect of
needed medical care, sexual discrimination, and shunning/shaming, among
others. The authors generated items by drawing from publicly available
measures of similar constructs and by writing new items to maximize
construct coverage. Each item they adapted to measure either a
potentially spiritually abusive event (inflicting emotional abuse by a
religious leader or group and/or with a religious or spiritual
component, usually involving coercion or control) or a theorized effect
of spiritual abuse, based on the reports of self-described survivors of
spiritual abuse in the qualitative literature. In developing items, the
authors sought to balance construct coverage with alternate wording and
efficiency for participants. They wanted to cover as much of the domain
as possible in a survey that could be completed in 15 minutes.

\hypertarget{participants-and-procedures}{%
\section*{Participants and
Procedures}\label{participants-and-procedures}}
\addcontentsline{toc}{section}{Participants and Procedures}

Participants in the initial SHAS validation study were adults who
responded to an online survey. To qualify for the survey, participants
had to be at least 18 years of age and to have identified themselves as
Christian in the past (Koch and Edstrom 2022).

The second author used snowball sampling to recruit participants to the
study. He promoted the survey through various podcast feeds (in audio
form during a podcast episode), e-mail lists, Facebook groups, and other
online groups associated with progressive Christian podcasts based in
the United States. Listeners/readers were encouraged to invite others
who satisfied the criteria to take part in the survey. The second author
also posted the link on his own social media accounts with a similar
encouragement to invite others who satisfied the study criteria. The
sampling priority was to reach the largest, most inclusive, community
sample. Participants were consented before beginning the survey. The
study was approved by the IRB at Northwest University. Data collection
began on January 28, 2021 and concluded on February 27, 2021.

\hypertarget{measures}{%
\section*{Measures}\label{measures}}
\addcontentsline{toc}{section}{Measures}

\hypertarget{demographics}{%
\subsection*{Demographics}\label{demographics}}
\addcontentsline{toc}{subsection}{Demographics}

Demographic variables included gender, sexual orientation, race, raised
in a Christian home, denomination, current religious identity, age, and
theological self-description.

\hypertarget{spiritual-harm-and-abuse-items}{%
\subsection*{Spiritual Harm and Abuse
Items}\label{spiritual-harm-and-abuse-items}}
\addcontentsline{toc}{subsection}{Spiritual Harm and Abuse Items}

The final survey instrument included a total of 66 prompts separated
into two distinct categories (Koch and Edstrom 2022). The first section,
External Events, included 52 prompts about potentially abusive
experiences. Prompts in this section measured prevalence by asking ``How
often have you experienced the above in a church or Christian group
setting?'' and providing participants a Liker scale of five response
categories: (1) Never, (2) Once or twice, (3) Sometimes, (4) Often, and
(5) All the time. The large number of prompts in this first section were
presented in a randomized order so that all prompts would receive
responses by approximately the same number of participants even if many
participants did not complete all of the prompts in this section.

The second section, Internal States, included 14 prompts about personal
feelings resulting from the abusive experiences described in the first
section. Prompts in this section measured prevalence by asking ``At any
point, how often have you experienced the above as a result of negative
religious experiences?'' Participants were provided the same 5-point
Likert scale of response categories as in the first section: (1) Never,
(2) Once or twice, (3) Sometimes, (4) Often, and (5) All the time.

\hypertarget{data-analysis}{%
\section*{Data Analysis}\label{data-analysis}}
\addcontentsline{toc}{section}{Data Analysis}

In their initial validation of the SHAS, Koch and Edstrom (2022)
examined the dimensionality of the 66 items. The authors used
exploratory factor analysis (EFA) to determine the factor structure of
the items. In these EFA analyses, the items loaded onto six
inter-related factors of eigenvalues of at least 1. This enabled the
authors to distill the item pool from 66 to 27 items and to identify
several meaningful subscales for potential future use by clinicians and
researchers.

For this study, we focused on the commonality of the items in order to
use item response theory to establish a single interval scale of
spiritual abuse. Item response theory models make two assumptions: (1) a
single latent trait explains responses to items (unidimensionality) and
(2) after controlling for this latent trait, items are weakly correlated
(local independence) (Embretson and Reise 2000; Reise and Waller 2009;
Reise, Bonifay, and Haviland 2013). The SHAS items were multidimensional
by design. The construct of spiritual harm and abuse is not conceptually
narrow, and to cover the breadth of this construct the Koch wrote two
distinct item sets. However, one can still appropriately fit a
unidimensional IRT model to multidimensional data (Reise, Moore, and
Haviland 2013). In this study we hypothesized a general factor of
spiritual abuse as the primary cause of responses to the full set of 66
SHAS items. We therefore sought evidence of unidimensionality and local
independence in order to calibrate the items and participants to a
single IRT scale.

We began by examining classical item statistics (skew, kurtosis,
item-total correlation) to identify poorly functioning items. Then we
conducted several analyses of dimensionality in search of sufficient
evidence of unidimensionality and local independence. This included
principal components analysis to compare the eigenvalues of the largest
and second largest components. We estimated a Rasch model and calculated
the proportion of variance in the data it explained. This we followed
with a Principal Components Analysis of Residuals (PCAR) to gauge the
importance of components in the residuals. We concluded by estimating
the Q3 matrix of correlations among item residuals.

We estimated a Rasch Rating Scale model to the SHAS items. We began by
examining plots of the category characteristic curves of the items to
visualize step parameters. Then we turned to item severity parameters
and item fit statistics to inspect the technical quality of the items.
We examined an item-person map to understand the fit of the items to the
participants. We concluded by assessing the fit of the model to the
data.

Finally, we conducted differential item functioning (DIF) analysis of
the items to examine item bias by gender, and age, and race. {[}We
applied logistic ordinal regression with IRT scoring. We used the
Chi-squared likelihood-ratio statistic as the initial DIF detection
criteria (alpha \textless{} 0.01), and a cut-off of McFadden pseudo R2Δ
≥ 0.02 in model comparisons to determine substantial DIF, a reasonable
threshold used in the development of self-reported health outcomes.{]}

We conducted all analyses in \textbf{R} (R Core Team 2022). We used the
\textbf{psych} package (Revelle 2022) and \textbf{eRm} (Mair, Hatzinger,
and Maier 2021) for analyses of dimensionality, the \textbf{mirt}
(Chalmers 2022) and \textbf{TAM} (Robitzsch, Kiefer, and Wu 2022)
packages for IRT analyses, and the \textbf{lordif} (Choi, Laura E.
Gibbons, and Crane 2016) package for DIF analyses.

\bookmarksetup{startatroot}

\hypertarget{results}{%
\chapter*{Results}\label{results}}
\addcontentsline{toc}{chapter}{Results}

\hypertarget{participant-characteristics}{%
\section*{Participant
Characteristics}\label{participant-characteristics}}
\addcontentsline{toc}{section}{Participant Characteristics}

In total, 3,222 individuals responded to the survey. Participants who
did not opt out and completed at least 50\% of the spiritual abuse item
pool were considered complete responders and included in the study
(N=3,219). Non-responders included those who responded to less than half
of the items (N=3).

Almost all participants (N = 3,064) answered demographics items, and the
results are presented Table @ref(tab:dems-table) below. Compared to the
general U.S. population, the sample was largely white (86\%) and
Protestant (57\%). The average respondent saw themselves as slightly
theologically liberal (M = 4.5, range 1-7). About 4 in 5 identified as
straight (82\%), and most were raised in a Christian home (87\%).

\begin{table}[!h]

\caption{Participant Characteristics}
\centering
\resizebox{\linewidth}{!}{
\begin{tabular}[t]{lcccccccccccc}
\toprule
  & n & mean & sd & median & trimmed & mad & min & max & range & skew & kurtosis & se\\
\midrule
Age & 3071 & 2.24 & 1.21 & 2 & 2.10 & 1.48 & 1 & 5 & 4 & 0.62 & -0.62 & 0.02\\
Gender & 3056 & 38.57 & 12.27 & 36 & 37.21 & 10.38 & 16 & 88 & 72 & 0.98 & 0.59 & 0.22\\
Sexual orientation & 3060 & 1.67 & 0.82 & 2 & 1.59 & 0.00 & 1 & 6 & 5 & 2.67 & 10.78 & 0.01\\
Race & 3064 & 1.36 & 1.01 & 1 & 1.07 & 0.00 & 1 & 6 & 5 & 3.08 & 9.04 & 0.02\\
Raised in Christian home & 3060 & 3.08 & 0.79 & 3 & 3.00 & 0.00 & 1 & 7 & 6 & 3.33 & 16.84 & 0.01\\
\addlinespace
Denomination of most abuse & 3061 & 1.09 & 0.28 & 1 & 1.00 & 0.00 & 1 & 2 & 1 & 2.93 & 6.60 & 0.00\\
Racial makeup of abusive community & 3047 & 5.20 & 3.33 & 5 & 4.69 & 2.96 & 1 & 17 & 16 & 1.47 & 2.19 & 0.06\\
Racial makeup of abusive group leadership & 3042 & 1.15 & 0.41 & 1 & 1.04 & 0.00 & 1 & 3 & 2 & 2.69 & 6.85 & 0.01\\
Gender makeup of abusive group leadership & 3044 & 1.09 & 0.35 & 1 & 1.00 & 0.00 & 1 & 3 & 2 & 3.90 & 15.41 & 0.01\\
Involvement in abusive group & 3048 & 1.82 & 0.79 & 2 & 1.76 & 1.48 & 1 & 5 & 4 & 0.48 & -0.78 & 0.01\\
\addlinespace
Current religious identification & 3054 & 5.54 & 1.99 & 7 & 5.92 & 0.00 & 1 & 7 & 6 & -1.25 & 0.24 & 0.04\\
Current theological identification & 3057 & 3.24 & 2.96 & 1 & 2.92 & 0.00 & 1 & 8 & 7 & 0.70 & -1.36 & 0.05\\
Current view of Bible & 2030 & 4.49 & 1.55 & 5 & 4.54 & 1.48 & 1 & 7 & 6 & -0.30 & -0.57 & 0.03\\
Generation & 3054 & 2.23 & 0.54 & 2 & 2.23 & 0.00 & 1 & 3 & 2 & 0.11 & -0.23 & 0.01\\
Podcast Use & 3056 & 2.43 & 0.79 & 2 & 2.35 & 0.00 & 1 & 5 & 4 & 0.76 & 0.22 & 0.01\\
\bottomrule
\end{tabular}}
\end{table}

\hypertarget{initial-item-analysis}{%
\section*{Initial Item Analysis}\label{initial-item-analysis}}
\addcontentsline{toc}{section}{Initial Item Analysis}

In Table X we report classical statistics for the 66 SHAS items. We
sorted the items by their mean. We flagged six items due to extreme
values of skew (less than -2 or greater than +2) or kurtosis (less than
-7 or greater than +7) and item-total correlations below 0.50. These
included items EQ17, EQ27, EQ41, EQ46, EQ55, and EQ60. Category response
proportions show that most people responded ``Never'' to these items.
Thus these items do not apply to the vast majority of people in this
population. We removed them from the item pool.

\begin{longtable}[t]{>{\raggedright\arraybackslash}p{2cm}cccccccccccc}
\toprule
\multicolumn{1}{c}{ } & \multicolumn{5}{c}{Category response proportion} & \multicolumn{7}{c}{ } \\
\cmidrule(l{3pt}r{3pt}){2-6}
  & 1 & 2 & 3 & 4 & 5 & n & mean & sd & skew & kurtosis & Item-total correlation & flag\\
\midrule
\endfirsthead
\multicolumn{13}{@{}l}{\textit{(continued)}}\\
\toprule
  & 1 & 2 & 3 & 4 & 5 & n & mean & sd & skew & kurtosis & Item-total correlation & flag\\
\midrule
\endhead

\endfoot
\bottomrule
\endlastfoot
EQ46 - Treated as less than because of my skin color & 0.94 & 0.03 & 0.02 & 0.01 & 0.00 & 3211 & 1.11 & 0.48 & 5.15 & 28.93 & 0.16 & *\\
EQ41 - Denied opportunities to serve because of my sexual orientation & 0.89 & 0.02 & 0.03 & 0.02 & 0.03 & 3211 & 1.27 & 0.88 & 3.30 & 9.87 & 0.31 & *\\
EQ17 - Treated as less than b/c my sexual orientation & 0.88 & 0.02 & 0.03 & 0.03 & 0.04 & 3208 & 1.34 & 0.97 & 2.87 & 6.96 & 0.35 & *\\
EQ47 - Medical care being postponed/withheld for religious reasons & 0.70 & 0.17 & 0.09 & 0.02 & 0.01 & 3218 & 1.48 & 0.86 & 1.93 & 3.40 & 0.52 & \\
EQ60 - Encouraged by leader to stay in abusive marriage & 0.76 & 0.09 & 0.07 & 0.05 & 0.02 & 3214 & 1.49 & 0.99 & 2.05 & 3.22 & 0.48 & *\\
\addlinespace
EQ55 - Hearing cultural references in sermons unfamiliar to my race/eth. subculture & 0.66 & 0.18 & 0.11 & 0.04 & 0.02 & 3205 & 1.58 & 0.95 & 1.66 & 2.12 & 0.28 & *\\
IQ72 - Feeling as if God harmed me directly & 0.64 & 0.17 & 0.13 & 0.04 & 0.02 & 3070 & 1.64 & 0.99 & 1.52 & 1.55 & 0.54 & \\
IQ81 - Nightmares about my negative religious experiences & 0.56 & 0.21 & 0.15 & 0.06 & 0.02 & 3073 & 1.77 & 1.04 & 1.23 & 0.66 & 0.65 & \\
EQ65 - Cut off/shunned by more religious family members & 0.59 & 0.16 & 0.13 & 0.08 & 0.04 & 3220 & 1.82 & 1.18 & 1.27 & 0.48 & 0.61 & \\
EQ10 - Asked to give up personal goals by pastor & 0.59 & 0.16 & 0.13 & 0.08 & 0.04 & 3220 & 1.83 & 1.18 & 1.25 & 0.44 & 0.64 & \\
\addlinespace
EQ26 - Pressured to forgive abuser while abuse was ongoing & 0.60 & 0.14 & 0.11 & 0.09 & 0.06 & 3212 & 1.87 & 1.27 & 1.24 & 0.21 & 0.66 & \\
EQ24 - Expected to consult pastor/leader before making non-religious decisions & 0.56 & 0.16 & 0.16 & 0.08 & 0.04 & 3221 & 1.89 & 1.18 & 1.10 & 0.07 & 0.66 & \\
EQ23 - Prayer replacing needed medical interventions & 0.51 & 0.21 & 0.17 & 0.08 & 0.03 & 3219 & 1.92 & 1.13 & 1.02 & 0.02 & 0.64 & \\
EQ27 - Denied opportunities to serve b/c of my gender & 0.60 & 0.11 & 0.12 & 0.09 & 0.08 & 3217 & 1.94 & 1.35 & 1.14 & -0.13 & 0.48 & *\\
EQ31 - Shunned/ignored by pastor/church/group & 0.46 & 0.22 & 0.18 & 0.09 & 0.05 & 3217 & 2.03 & 1.19 & 0.92 & -0.20 & 0.64 & \\
\addlinespace
EQ30 - Deterred from seeking mental health treatment/counseling/medication & 0.50 & 0.17 & 0.17 & 0.11 & 0.05 & 3219 & 2.04 & 1.25 & 0.88 & -0.46 & 0.70 & \\
EQ43 - Shamed by pastor/group for poor spiritual/moral performance & 0.46 & 0.22 & 0.18 & 0.10 & 0.04 & 3217 & 2.04 & 1.18 & 0.86 & -0.31 & 0.73 & \\
IQ68 - Anxiety attacks triggered by religious stimuli & 0.49 & 0.17 & 0.19 & 0.11 & 0.05 & 3073 & 2.06 & 1.23 & 0.83 & -0.51 & 0.68 & \\
EQ52 - Blamed for harm I suffered rather than blaming who harmed me & 0.48 & 0.19 & 0.16 & 0.12 & 0.05 & 3217 & 2.08 & 1.26 & 0.85 & -0.52 & 0.76 & \\
IQ70 - Feeling betrayed by God & 0.42 & 0.24 & 0.21 & 0.09 & 0.05 & 3069 & 2.11 & 1.18 & 0.79 & -0.36 & 0.56 & \\
\addlinespace
EQ49 - Leadership/group protecting and elevating abusive individuals & 0.44 & 0.22 & 0.18 & 0.11 & 0.05 & 3213 & 2.12 & 1.23 & 0.79 & -0.50 & 0.72 & \\
EQ16 - Church/community abandoning me in difficult time & 0.45 & 0.21 & 0.17 & 0.11 & 0.06 & 3217 & 2.13 & 1.27 & 0.83 & -0.50 & 0.69 & \\
EQ57 - Developing mental/physical ailments from conforming to group/leader’s expectation & 0.48 & 0.16 & 0.16 & 0.12 & 0.07 & 3219 & 2.15 & 1.33 & 0.78 & -0.70 & 0.72 & \\
EQ20 - Members pressured to give money despite financial hardship & 0.50 & 0.15 & 0.15 & 0.11 & 0.09 & 3217 & 2.16 & 1.39 & 0.82 & -0.71 & 0.68 & \\
EQ62 - Scripture used to justify physical violence & 0.43 & 0.19 & 0.21 & 0.11 & 0.05 & 3220 & 2.16 & 1.24 & 0.69 & -0.67 & 0.65 & \\
\addlinespace
EQ59 - Pastor/group blame victim for their abuse & 0.42 & 0.21 & 0.19 & 0.12 & 0.05 & 3217 & 2.16 & 1.23 & 0.70 & -0.65 & 0.76 & \\
EQ21 - Taught I would risk Hell if left my church/group & 0.51 & 0.13 & 0.13 & 0.10 & 0.13 & 3218 & 2.20 & 1.47 & 0.81 & -0.83 & 0.69 & \\
EQ50 - Treated as less than because of my gender & 0.52 & 0.08 & 0.15 & 0.13 & 0.11 & 3219 & 2.23 & 1.47 & 0.70 & -1.04 & 0.54 & \\
IQ82 - Having trouble navigating life outside my church/community & 0.38 & 0.23 & 0.24 & 0.11 & 0.05 & 3071 & 2.24 & 1.21 & 0.62 & -0.62 & 0.61 & \\
EQ64 - Witnessing women pressured to stay in unfaithful/abusive marriages & 0.39 & 0.23 & 0.19 & 0.12 & 0.07 & 3220 & 2.26 & 1.27 & 0.67 & -0.70 & 0.71 & \\
\addlinespace
EQ28 - Leadershi/group protecting abusive individuals & 0.37 & 0.24 & 0.19 & 0.13 & 0.06 & 3216 & 2.26 & 1.25 & 0.63 & -0.73 & 0.72 & \\
EQ40 - Threatening Divine punishment to keep group members in line & 0.44 & 0.17 & 0.18 & 0.12 & 0.10 & 3220 & 2.28 & 1.39 & 0.67 & -0.88 & 0.75 & \\
EQ63 - Shamed by pastor/group for raising questions or concerns & 0.36 & 0.23 & 0.22 & 0.13 & 0.06 & 3217 & 2.30 & 1.25 & 0.58 & -0.77 & 0.77 & \\
EQ19 - Behavior excessively monitored by pastor/group & 0.40 & 0.18 & 0.19 & 0.15 & 0.08 & 3219 & 2.32 & 1.33 & 0.56 & -0.96 & 0.75 & \\
EQ51 - Extreme pressure to be pastor/missionary/spiritual leader & 0.41 & 0.19 & 0.17 & 0.14 & 0.10 & 3216 & 2.33 & 1.38 & 0.59 & -0.98 & 0.62 & \\
\addlinespace
EQ53 - Scripture used to justify abusive parent-child behavior & 0.39 & 0.19 & 0.20 & 0.14 & 0.09 & 3218 & 2.36 & 1.35 & 0.55 & -0.97 & 0.73 & \\
IQ79 - Distrust of God & 0.33 & 0.23 & 0.24 & 0.12 & 0.08 & 3068 & 2.40 & 1.27 & 0.50 & -0.81 & 0.61 & \\
EQ34 - Scripture used to justify physical punishment/severe discipline & 0.34 & 0.19 & 0.22 & 0.16 & 0.09 & 3219 & 2.46 & 1.33 & 0.42 & -1.05 & 0.65 & \\
EQ29 - Feeling special when in pastor’s good graces; otherwise ignored & 0.34 & 0.19 & 0.22 & 0.16 & 0.09 & 3213 & 2.47 & 1.34 & 0.40 & -1.08 & 0.69 & \\
IQ73 - Self-hatred or self-loathing & 0.31 & 0.21 & 0.24 & 0.15 & 0.09 & 3072 & 2.50 & 1.31 & 0.38 & -1.00 & 0.64 & \\
\addlinespace
EQ11 - Disagree w/pastor portrayed as evil & 0.32 & 0.21 & 0.20 & 0.17 & 0.10 & 3216 & 2.52 & 1.35 & 0.38 & -1.12 & 0.77 & \\
EQ58 - Terror/horror used to motivate religious decisions & 0.32 & 0.18 & 0.24 & 0.16 & 0.10 & 3220 & 2.53 & 1.34 & 0.34 & -1.09 & 0.73 & \\
EQ7 - Pastor speaking on God's behalf & 0.35 & 0.18 & 0.19 & 0.16 & 0.13 & 3215 & 2.54 & 1.43 & 0.39 & -1.21 & 0.67 & \\
EQ42 - Seeing others shamed/shunned by pastor/leader/group & 0.25 & 0.24 & 0.26 & 0.17 & 0.08 & 3215 & 2.60 & 1.25 & 0.28 & -0.96 & 0.77 & \\
EQ44 - Love/acceptance offered only for high spiritual/moral performance & 0.30 & 0.18 & 0.23 & 0.19 & 0.10 & 3217 & 2.61 & 1.35 & 0.24 & -1.20 & 0.80 & \\
\addlinespace
EQ56 - Being made to feel I was crazy/weird for doubts/questions & 0.27 & 0.17 & 0.23 & 0.21 & 0.12 & 3217 & 2.72 & 1.37 & 0.14 & -1.24 & 0.82 & \\
EQ18 - Church/pastor discourage critical thinking & 0.27 & 0.17 & 0.24 & 0.19 & 0.13 & 3220 & 2.75 & 1.37 & 0.14 & -1.21 & 0.77 & \\
EQ45 - Developmentally-inappropriate/anxious Hell/Satan/demons taught to young children & 0.29 & 0.17 & 0.20 & 0.18 & 0.16 & 3219 & 2.76 & 1.45 & 0.18 & -1.33 & 0.72 & \\
IQ78 - A lack of self-worth & 0.23 & 0.19 & 0.27 & 0.21 & 0.11 & 3071 & 2.77 & 1.30 & 0.10 & -1.09 & 0.67 & \\
EQ36 - Mental/physical problems interpreted as spiritual/moral weakness & 0.24 & 0.19 & 0.24 & 0.21 & 0.12 & 3216 & 2.78 & 1.34 & 0.11 & -1.17 & 0.77 & \\
\addlinespace
EQ61 - Developmentally-inappropriate/anxious end times descriptions taught to young children & 0.30 & 0.15 & 0.20 & 0.19 & 0.17 & 3219 & 2.78 & 1.47 & 0.14 & -1.37 & 0.72 & \\
EQ32 - Feeling unable to express unhappiness & 0.24 & 0.16 & 0.27 & 0.21 & 0.11 & 3219 & 2.79 & 1.32 & 0.05 & -1.14 & 0.73 & \\
IQ74 - Sadness over the loss of my faith/religious community & 0.22 & 0.19 & 0.27 & 0.21 & 0.11 & 3071 & 2.79 & 1.29 & 0.07 & -1.08 & 0.60 & \\
IQ76 - Feeling I wasted years of my life in a church/set of beliefs & 0.28 & 0.16 & 0.19 & 0.18 & 0.19 & 3072 & 2.83 & 1.48 & 0.12 & -1.39 & 0.74 & \\
EQ54 - Being explicitly taught to distrust my intuitions & 0.23 & 0.16 & 0.24 & 0.20 & 0.17 & 3217 & 2.92 & 1.40 & 0.01 & -1.26 & 0.75 & \\
\addlinespace
EQ37 - Made to feel less spiritually mature than pastor/leadership & 0.22 & 0.17 & 0.24 & 0.21 & 0.16 & 3221 & 2.94 & 1.38 & -0.01 & -1.22 & 0.75 & \\
EQ38 - Unrealistic demands placed on my moral/religious behavior & 0.22 & 0.16 & 0.24 & 0.20 & 0.18 & 3217 & 2.95 & 1.40 & -0.01 & -1.26 & 0.80 & \\
IQ71 - Avoiding religious activities/settings to reduce distressing feelings & 0.19 & 0.17 & 0.24 & 0.22 & 0.17 & 3074 & 3.01 & 1.36 & -0.06 & -1.19 & 0.74 & \\
IQ69 - Lack of spiritual direction/purpose & 0.13 & 0.20 & 0.32 & 0.23 & 0.12 & 3071 & 3.01 & 1.20 & -0.06 & -0.83 & 0.57 & \\
EQ39 - Made to feel shame over natural sexual desires (not actions) & 0.24 & 0.13 & 0.20 & 0.21 & 0.22 & 3219 & 3.05 & 1.48 & -0.12 & -1.37 & 0.72 & \\
\addlinespace
IQ80 - Feeling isolated & 0.14 & 0.18 & 0.29 & 0.27 & 0.13 & 3070 & 3.07 & 1.23 & -0.17 & -0.91 & 0.68 & \\
IQ77 - Anger upon reflecting on negative religious experiences & 0.15 & 0.18 & 0.26 & 0.26 & 0.15 & 3075 & 3.08 & 1.28 & -0.16 & -1.02 & 0.77 & \\
EQ35 - Taught to distrust my emotions & 0.19 & 0.14 & 0.23 & 0.23 & 0.21 & 3217 & 3.12 & 1.39 & -0.18 & -1.20 & 0.74 & \\
EQ25 - Feeling unable to raise questions and issues & 0.15 & 0.15 & 0.25 & 0.27 & 0.18 & 3217 & 3.18 & 1.32 & -0.26 & -1.04 & 0.76 & \\
EQ66 - Others treated as less than due to their sexual orientation & 0.18 & 0.12 & 0.21 & 0.25 & 0.24 & 3216 & 3.27 & 1.41 & -0.34 & -1.15 & 0.71 & \\
\addlinespace
EQ22 - Expected to follow pastor/leader rules re: dating/marriage/sex & 0.20 & 0.11 & 0.15 & 0.24 & 0.30 & 3218 & 3.33 & 1.49 & -0.39 & -1.28 & 0.70 & \\*
\end{longtable}

\hypertarget{unidimensionality-and-local-independence}{%
\section*{Unidimensionality and Local
Independence}\label{unidimensionality-and-local-independence}}
\addcontentsline{toc}{section}{Unidimensionality and Local Independence}

\textbf{Principal Components}. We first analyzed the principal
components in the SHAS items. Like factor analysis, component analysis
attempts to summarize the common variance in items by identifying a
smaller set of latent sources of covariation. Principal components
analysis differs from factor analysis by treating all of the common
variance among the items as the total variance to be explained {[}verify
this{]}.

\begin{figure}

{\centering \includegraphics{./results_files/figure-pdf/pca-run-1.pdf}

}

\caption{Principal Components Analysis}

\end{figure}

As shown in the scree plot, the first component was large, accounting
for nearly 30\% of the variance. The second, third, and fourth
components accounted for 4\%, 3\%, and 2\% of the variance,
respectively. The ratio of the eigenvalue of the first (largest)
component to that of the second is approximately 13.

\textbf{Proportion of variance}. We estimated a unidimensional Rasch
model on dichotomized responses to the 59 SHAS items and saved the
residuals of the person parameters. We calculated the variance in the
observed item responses and the variance of the residuals.

\begin{verbatim}
[1] 0.2264541
\end{verbatim}

The proportion of variance explained by the Rasch model was .23. If the
unidimensionalty assumption is safely met if the Rasch model explains
20\% of the variance in the data (Reckase 1979), then the SHAS data meet
this criterion.

\textbf{Principal Component Analysis of Residuals (PCAR)}. We examined
the principal components of the correlations among residuals of the
Rasch analysis. The premise is that once the Rasch model has been
estimated, correlations among the item residuals should be minimal.
Linacre () suggests that contrasts with eigenvalues of 2.0 or below can
be considered noise.

\begin{figure}

{\centering \includegraphics{./results_files/figure-pdf/uni-pcar-1.pdf}

}

\caption{Contrasts from PCA of Standardized Residual Correlations}

\end{figure}

In Figure @ref(fig:uni-pcar), our PCAR analysis found only one contrast
that rose above that 2.0 threshold.

\textbf{Q3}. Finally, we calculated the Q3 statistic (Yen, 1984). This
statistic examines the correlations among item residuals, the premise
being that the latent trait should account for so much common variance
in the item responses that any net correlations among the items should
be weak. The Q3 statistic index criteria are that the raw residual
correlation between pairs of items should never exceed 0.10 (Marais \&
Andrich, 2008). Of the 1,711 item pairs, the mean correlation was -0.017
with a standard deviation of 0.035. Thus, most of the residual
correlations among items were very weak.

\begin{verbatim}

WLE Reliability= 0.918 
\end{verbatim}

\begin{verbatim}
Yen's Q3 Statistic based on an estimated theta score 
*** 60 Items | 1770 item pairs
*** Q3 Descriptives
     M     SD    Min    10%    25%    50%    75%    90%    Max 
-0.017  0.036 -0.109 -0.055 -0.040 -0.020 -0.001  0.025  0.321 
\end{verbatim}

\begin{figure}

{\centering \includegraphics{./results_files/figure-pdf/q3-plot-1.pdf}

}

\caption{Matrix of Correlations of Item Residuals}

\end{figure}

Figure @ref(fig:q3-plot) plots the matrix of correlations between item
residuals. Each square depicts a correlation. The squares are shaded in
grey with the lightest shade indicating the weakest correlations. As
hoped, the vast majority of squares are very light, indicating weak
correlations among the item residuals.

\hypertarget{rasch-rating-scale-analysis}{%
\section*{Rasch Rating Scale
Analysis}\label{rasch-rating-scale-analysis}}
\addcontentsline{toc}{section}{Rasch Rating Scale Analysis}

We used the Rasch Rating Scale Model (RSM) (Andrich 1978) to examine the
SHAS item pool. The RSM is a Rasch model (Rasch 1960) that combines (or
``calibrates'') information from items with information from persons to
arrive at a common scale for measuring both an item's and a person's
level of a latent trait, in this case, severity of spiritual abuse. This
scale is expressed in logits (log odds units). Whereas the Rasch model
was designed for use with dichotomous items (0 = incorrect response, 1 =
correct response), the RSM is an extension for use with polytomous items
based on rating scales such as Likert scales of agreement (Embretson and
Reise 2000).

We also use different language from conventional IRT analyses. Because
IRT was developed to measure student achievement, IRT statistics for
both items and persons tend to use language of ``ability'' and
``difficulty.'' For our purpose to measure a psychological trait such as
the experience of spiritual abuse, we chose instead to adopt language of
``severity.'' We used person parameters and scale scores to represent
participants' severity of abuse and to describe item parameters as
severity parameters.

The RSM model estimates the probability of a person choosing among
several response options (i.e.~Never, Once or twice) given two values:
the severity of abuse represented by the item, and the person's severity
of the latent trait (spiritual abuse). The increase in severity involved
in choosing between two response categories is called a ``step
parameter''. The RSM assumes that all items use the same rating scale
and that these same response categories distinguish persons equally well
for each item. For these reasons the RSM estimates the same step
parameters for all the items.

To convey this information visually, Figure @ref(fig:rsm-more-plots)
shows plots of two items. In both, the horizontal axis is the scale of
spiritual abuse expressed in logits ranging from -2 to +2. The vertical
axis is the probability of selecting a given response (i.e., ``Always'')
given the person's overall severity of spiritual abuse. The blue line is
the probability of selecting Category 1 (``Never'') for persons with
less severe spiritual abuse. Persons with less severe spiritual abuse
are most likely to select ``Never'' while those with more severe abuse
are most likely to select ``Always.'' Between these two extremes are
persons with more moderate or average spiritual abuse. These persons are
more likely to Categories 3 or 4, ``Sometimes'' or ``Often'',
respectively. Both plots show curves that display identical step
parameters.

The plots point out a step reversal between ``Never'' and ``Once or
twice''. Persons reporting the least severe spiritual abuse were most
likely to report ``Never'' to this prompt of medical care being
postponed or withheld for religious reasons. Persons farther along the
scale who have experienced ``average'' average spiritual abuse should be
more likely to report ``Once or twice'' but were still more likely to
report ``Never''. This finding suggests the ``Never'' and ``Once or
twice'' categories might be collapsed.

\begin{verbatim}
Iteration in WLE/MLE estimation  1   | Maximal change  2.7551 
Iteration in WLE/MLE estimation  2   | Maximal change  0.8593 
Iteration in WLE/MLE estimation  3   | Maximal change  0.4864 
Iteration in WLE/MLE estimation  4   | Maximal change  0.1926 
Iteration in WLE/MLE estimation  5   | Maximal change  0.028 
Iteration in WLE/MLE estimation  6   | Maximal change  0.0012 
Iteration in WLE/MLE estimation  7   | Maximal change  0 
----
 WLE Reliability= 0.963 
\end{verbatim}

\includegraphics{./results_files/figure-pdf/rsm-more-plots-1.pdf}

\includegraphics{./results_files/figure-pdf/rsm-more-plots-2.pdf}

\textbf{Item Parameters and Fit Statistics}. Table
@ref(tab:rsm-item-tbl) reports item parameters and fit statistics for
the SHAS items. The first column, xsi, reports the severity parameter of
the item, and we have sorted the items by this value from least to most
severe to facilitate comparison between items.

Table @ref(tab:rsm-item-tbl) also reports fit statistics for the items.
These statistics reflect how closely the observed patterns of item
responses fit the patterns of item responses predicted by the RSM. These
fit statistics are chi-square statistics which examine the cumulative
difference the observed pattern of item responses and the pattern of
item responses that the model would expect.

Two fit statistics commonly used in IRT models are the infit mean square
and the outfit mean square (Bond and Fox 2015). The infit statistic
places greater emphasis on unexpected responses that are close to the
persons and item location. The outfit is sensitive to unexpected
responses that are far from the location. The expected value of infit or
outfit for each item is 1.0, with a range of acceptable values ranging
from 0.5 to 1.5. Values outside these boundaries indicate a lack of fit
between items and the model. All but one of the 60 items had infit and
outfit statistics within the acceptable range.

\begin{longtable}[t]{>{\raggedright\arraybackslash}p{11cm}ccccc}
\caption{Estimated Item Parameters for the Rating Scale Model and Item Chi-Square Fit Statistics}\\
\toprule
\multicolumn{2}{c}{ } & \multicolumn{2}{c}{Item Parameters} & \multicolumn{2}{c}{Fit Statistics} \\
\cmidrule(l{3pt}r{3pt}){3-4} \cmidrule(l{3pt}r{3pt}){5-6}
  & subscale & xsi & se.xsi & Outfit & Infit\\
\midrule
\endfirsthead
\caption[]{Estimated Item Parameters for the Rating Scale Model and Item Chi-Square Fit Statistics \textit{(continued)}}\\
\toprule
  & subscale & xsi & se.xsi & Outfit & Infit\\
\midrule
\endhead

\endfoot
\bottomrule
\endlastfoot
*EQ22 - Expected to follow pastor/leader rules re: dating/marriage/sex & controlling leadership & -0.4214607 & 0.0188285 & 1.2082569 & 1.2593340\\
EQ66 - Others treated as less than due to their sexual orientation &  & -0.3532077 & 0.0187160 & 1.0954200 & 1.0454220\\
EQ25 - Feeling unable to raise questions and issues &  & -0.2606833 & 0.0185820 & 0.7650711 & 0.7559072\\
EQ35 - Taught to distrust my emotions &  & -0.1933913 & 0.0185025 & 1.0621576 & 0.9718729\\
*IQ77 - Anger upon reflecting on negative religious experiences & internal distress & -0.1274391 & 0.0188390 & 0.7110960 & 0.6893819\\
\addlinespace
EQ39 - Made to feel shame over natural sexual desires (not actions) &  & -0.1152678 & 0.0184252 & 1.1124514 & 1.1235593\\
*IQ80 - Feeling isolated & internal distress & -0.1146598 & 0.0188409 & 0.9442974 & 0.8695395\\
*IQ69 - Lack of spiritual direction/purpose & internal distress & -0.0540643 & 0.0187893 & 1.3368459 & 1.1180008\\
*IQ71 - Avoiding religious activities/settings to reduce distressing feelings & internal distress & -0.0445240 & 0.0187747 & 0.9249188 & 0.9228522\\
EQ38 - Unrealistic demands placed on my moral/religious behavior &  & -0.0028530 & 0.0183624 & 0.7463425 & 0.7728259\\
\addlinespace
EQ37 - Made to feel less spiritually mature than pastor/leadership &  & 0.0081963 & 0.0183481 & 0.9160947 & 0.8967232\\
EQ54 - Being explicitly taught to distrust my intuitions &  & 0.0325563 & 0.0183543 & 0.9371381 & 0.9288786\\
IQ76 - Feeling I wasted years of my life in a church/set of beliefs &  & 0.1480538 & 0.0187216 & 1.0575000 & 1.0853447\\
EQ32 - Feeling unable to express unhappiness &  & 0.1739744 & 0.0183405 & 0.9249743 & 0.8863517\\
EQ61 - Developmentally-inappropriate/anxious end times descriptions taught to young children &  & 0.1804548 & 0.0183426 & 1.1043663 & 1.1403253\\
\addlinespace
EQ36 - Mental/physical problems interpreted as spiritual/moral weakness &  & 0.1826974 & 0.0183464 & 0.8355542 & 0.8236019\\
*IQ74 - Sadness over the loss of my faith/religious community & internal distress & 0.1879652 & 0.0187295 & 1.3272167 & 1.1759672\\
*EQ45 - Developmentally-inappropriate/anxious Hell/Satan/demons taught to young children & spiritual violence & 0.2034352 & 0.0183470 & 1.0823745 & 1.1054398\\
*IQ78 - A lack of self-worth & internal distress & 0.2045263 & 0.0187349 & 1.1031718 & 1.0264169\\
EQ18 - Church/pastor discourage critical thinking &  & 0.2177587 & 0.0183510 & 0.8205468 & 0.8390994\\
\addlinespace
EQ56 - Being made to feel I was crazy/weird for doubts/questions &  & 0.2429686 & 0.0183689 & 0.6762540 & 0.7087537\\
EQ44 - Love/acceptance offered only for high spiritual/moral performance &  & 0.3632040 & 0.0184507 & 0.7491525 & 0.7556204\\
EQ42 - Seeing others shamed/shunned by pastor/leader/group &  & 0.3818354 & 0.0184788 & 0.7445773 & 0.7290104\\
*EQ7 - Pastor speaking on God's behalf & controlling leadership & 0.4492462 & 0.0185514 & 1.3723504 & 1.2636929\\
*EQ58 - Terror/horror used to motivate religious decisions & spiritual violence & 0.4494840 & 0.0185360 & 0.9129435 & 0.9432259\\
\addlinespace
EQ11 - Disagree w/pastor portrayed as evil &  & 0.4692857 & 0.0185727 & 0.8290139 & 0.8524570\\
*IQ73 - Self-hatred or self-loathing & internal distress & 0.5069579 & 0.0189847 & 1.1905117 & 1.1612019\\
EQ29 - Feeling special when in pastor’s good graces; otherwise ignored &  & 0.5197064 & 0.0186501 & 1.1100359 & 1.0798013\\
*EQ34 - Scripture used to justify physical punishment/severe discipline & spiritual violence & 0.5317607 & 0.0186536 & 1.2024332 & 1.1701601\\
*IQ79 - Distrust of God & harmful God-image & 0.6207993 & 0.0191931 & 1.3348823 & 1.2033472\\
\addlinespace
EQ53 - Scripture used to justify abusive parent-child behavior &  & 0.6449712 & 0.0188664 & 0.9598614 & 1.0066135\\
*EQ51 - Extreme pressure to be pastor/missionary/spiritual leader & controlling leadership & 0.6744194 & 0.0189356 & 1.4108410 & 1.3815284\\
*EQ19 - Behavior excessively monitored by pastor/group & controlling leadership & 0.6934427 & 0.0189693 & 0.8533769 & 0.9242834\\
EQ63 - Shamed by pastor/group for raising questions or concerns &  & 0.7152483 & 0.0190264 & 0.7511239 & 0.7785983\\
EQ40 - Threatening Divine punishment to keep group members in line &  & 0.7402511 & 0.0190774 & 0.9203295 & 1.0187414\\
\addlinespace
*EQ28 - Leadershi/group protecting abusive individuals & maintain system & 0.7577330 & 0.0191333 & 0.9207000 & 0.9028342\\
EQ64 - Witnessing women pressured to stay in unfaithful/abusive marriages &  & 0.7669132 & 0.0191440 & 0.9969206 & 0.9909701\\
EQ50 - Treated as less than because of my gender & gender discrimination & 0.7946816 & 0.0192217 & 1.9534883 & 1.9178832\\
*IQ82 - Having trouble navigating life outside my church/community & internal distress & 0.8030296 & 0.0196156 & 1.1692310 & 1.1525204\\
EQ21 - Taught I would risk Hell if left my church/group &  & 0.8324140 & 0.0193279 & 1.3029521 & 1.3875107\\
\addlinespace
EQ59 - Pastor/group blame victim for their abuse &  & 0.8747606 & 0.0194649 & 0.7555056 & 0.8087351\\
*EQ62 - Scripture used to justify physical violence & spiritual violence & 0.8785696 & 0.0194647 & 1.0704151 & 1.1049062\\
EQ20 - Members pressured to give money despite financial hardship &  & 0.8849166 & 0.0194970 & 1.1366942 & 1.2719704\\
EQ57 - Developing mental/physical ailments from conforming to group/leader’s expectation &  & 0.8896199 & 0.0195048 & 0.9437715 & 1.0778017\\
*EQ16 - Church/community abandoning me in difficult time & maintain system & 0.9185121 & 0.0196025 & 1.0300395 & 1.0520437\\
\addlinespace
EQ49 - Leadership/group protecting and elevating abusive individuals &  & 0.9243923 & 0.0196385 & 0.8856635 & 0.9238426\\
*IQ70 - Feeling betrayed by God & harmful God-image & 0.9546912 & 0.0201015 & 1.3759280 & 1.2626165\\
*EQ52 - Blamed for harm I suffered rather than blaming who harmed me & maintain system & 0.9838622 & 0.0198359 & 0.8025177 & 0.8874390\\
IQ68 - Anxiety attacks triggered by religious stimuli &  & 1.0256093 & 0.0203489 & 0.9860538 & 1.0666330\\
EQ30 - Deterred from seeking mental health treatment/counseling/medication &  & 1.0260222 & 0.0199851 & 0.9575596 & 1.0295984\\
\addlinespace
EQ43 - Shamed by pastor/group for poor spiritual/moral performance &  & 1.0261756 & 0.0199917 & 0.8082705 & 0.8382411\\
*EQ31 - Shunned/ignored by pastor/church/group & maintain system & 1.0448962 & 0.0200638 & 1.1253225 & 1.0832325\\
EQ23 - Prayer replacing needed medical interventions &  & 1.1947445 & 0.0207157 & 1.0631138 & 1.0834253\\
*EQ24 - Expected to consult pastor/leader before making non-religious decisions & controlling leadership & 1.2385072 & 0.0209273 & 1.0174869 & 1.0758640\\
*EQ26 - Pressured to forgive abuser while abuse was ongoing & maintain system & 1.2605408 & 0.0210791 & 1.3778252 & 1.2962862\\
\addlinespace
EQ10 - Asked to give up personal goals by pastor &  & 1.3223656 & 0.0213670 & 1.1007863 & 1.1880974\\
EQ65 - Cut off/shunned by more religious family members &  & 1.3298044 & 0.0214183 & 1.2710367 & 1.2564825\\
IQ81 - Nightmares about my negative religious experiences &  & 1.4234865 & 0.0223585 & 0.8982022 & 0.9662203\\
*IQ72 - Feeling as if God harmed me directly & harmful God-image & 1.6368994 & 0.0238679 & 1.1400541 & 1.2181687\\
EQ47 - Medical care being postponed/withheld for religious reasons &  & 1.9375081 & 0.0260704 & 1.1272914 & 1.1571689\\*
\end{longtable}

\begin{figure}

{\centering \includegraphics{./results_files/figure-pdf/rsm-item-hist-1.pdf}

}

\caption{Distribution of Item Severity Parameters}

\end{figure}

\textbf{Reliability}. The RSM expresses internal consistency reliability
as Rasch person separation. Based on Table X, the SHAS has a high Rasch
person separation reliability value of .963, indicating that the
estimated RSM scale discriminated well between persons with varying
severity of spiritual abuse.

The SHAS also has a high Rasch item separation reliability with Rel =
.XX, χ2 (XX) = XXXX.X, p \textless{} .XX, implying that the items have a
good spread in terms of item ordering and hierarchy.

\begin{verbatim}
[1] 1
\end{verbatim}

\textbf{Item-Person Map}. Figure @ref(fig:rsm-wrightmap) displays an
item-person map (also called a ``Wright map'') that shows the item
severities of the SHAS items and the person severities for each person
who completed the survey. All these severities are estimated in logits
(log odds units) as the unit of measures. This means the higher the
logit value for an item is, the less likely it was for a person to
endorse that particular item. On the other hand, higher logit values for
each person indicate more severe spiritual abuse. On the variable map,
the mean item severity is constrained to be 0.00 with person severities
being relative to that mean item severity While column 1 shows the
latent continuum in terms of logit values as the unit of measurement
underlying the SHAS, columns 2 and 3 represent the severities of the
people and the items.

Figure @ref(fig:rsm-wrightmap) shows good overlap between the person
trait and item severity as evidenced by the match between the mean of
the person severity (M = − 0.58) and the mean of the item severity (M =
0.00). This suggests that the majority of the items were appropriate for
the sample. While person severity measures range from 2.13 logits to −
5.41 logits (M = − 0.58, SD = 0.79, N = 952), the item severities range
from 1.14 logits to − 1.08 logits (M = 0.00, SD= 0.47, N = 45). The item
representing the most severe spiritual abuse for participants is Item 9,
``I feel \ldots{} XXX'' with the severity at 1.14 logits (also reported
in Appendix Table 2). By contrast the item representing the least severe
spiritual abuse is item 10, ``I would \ldots{}'', with − 1.08 logits.

Regarding the variable map, the SHAS appears to have a few limitations.
First, the SHAS has a lack of items that provide information about
students who are located at the very high end of the continuum, above
1.14 logits, and those who are placed on the very low end of the
continuum, particularly below − 0.85 logits. Hence, the SHAS seems to be
not targeted at identifying people who have very high and low severity
of spiritual abuse. Second, in terms of the distributions of items with
respect to their underlying factors (see Fig. 1 and Appendix Table 2),
most of the items are spread sufficiently between 1.14 logits and − 1.08
logits except the items which belong to use of mathematics in daily
life, represented by seven items (i.e., items 9, 15, 17, 26, 29, 38,
45). That five of these seven items (i.e., items 9, 15, 17, 26, 29) are
located at the high end of the continuum indicates that these items are
the most difficult items for students to agree with. This, in turn,
suggests that these items do not provide information about students with
low or moderate levels of math anxiety related to the use of mathematics
in daily life on the high end of the continuum. Third, in terms of the
redundancy of the items that provide the exact same information, item 17
and item 26, represented by the use of mathematics in daily life; item 6
and item 37, represented by apprehension of math lessons; and item 21
and item 44, represented by test and evaluation anxiety, have the same
mean ratings, item difficulties, and standard errors as within pairs
(see Appendix Table 2). Therefore, item 17, item 37, and item 21, which
have slightly worse fit statistics (Infit and Outfit), can be released
from the scale to shorten the length of the scale's administration.

\begin{figure}

{\centering \includegraphics{./results_files/figure-pdf/rsm-wrightmap-1.pdf}

}

\caption{Item-Person Map}

\end{figure}

\textbf{Model-Data Fit}. To assess the overall fit of the model to the
item response data we use two statistics. The first is the Root Mean
Square Error of Approximation (RMSEA). The suggested cutoff for RMSEA is
.06. The second is the CFI. The suggested cutoff for the CFI is .95.
Model fit statistics for the SHAS items are presented in Table
@ref(tab:rsm-model-fit).

\begin{verbatim}
'log Lik.' -225573.6 (df=64)
\end{verbatim}

The obtained RMSEA value of .09 (95\% CI{[}.092, .093{]}) exceeds the
cutoff of .06. The CFI of .953 meets the recommended .95 threshold.
Together these statistics evidence that the RSM fits the SHAS items.

\hypertarget{dif-analysis}{%
\section*{DIF Analysis}\label{dif-analysis}}
\addcontentsline{toc}{section}{DIF Analysis}

Differential item functioning (DIF) examines how an item functions
differently for people of equal standing on the trait. In this case, we
conducted DIF analysis of the SHAS items to examine item bias by gender,
and age, and race. {[}We applied logistic ordinal regression with IRT
scoring. {[}We used the Chi-squared likelihood-ratio statistic as the
initial DIF detection criteria (alpha \textless{} 0.01) and a cut-off of
McFadden pseudo R2Δ ≥ 0.02 in model comparisons to determine substantial
DIF, a reasonable threshold used in the development of self-reported
health outcomes.{]}

In this analysis, we examined the degree to which there was evidence of
uniform differential item functioning (DIF) between two subgroups of
participants (group 1 and group 2). As a first step we examined item
severity estimates and standard errors specific to each subgroup using
the Rating Scale Model Andrich (1978). We estimated item severity for
the two subgroups using a combined analysis of both subgroups and then
estimating the group-specific item severities and standard errors are
shown in Table X.

To examine the differences in item severity between subgroups, we
calculated standardized differences following Wright and Masters (1982)
as follows:

{[}equation{]}

where\\
z is the standardized difference, d1 is the item severity specific to
Subgroup 1, d2 is the item severity specific to Subgroup 2,\\
s e 2 1 is the standard error of the item severity specific to Subgroup
1, and\\
s e 2 2 is the standard error of the item severity specific to Subgroup
2. Using the formulation of the z statistic, higher values of z indicate
higher item locations (more-severe to endorse) for Subgroup 1 compared
to Subgroup 2.

Figure X shows a plot of the z-statistics for the four items; these
values are also presented numerically in Table X. In the figure, the
x-axis shows the item identification numbers, and the y-axis shows the
value of the z-statistic. Boundaries at +2 and -2 are indicated using
dashed horizontal lines to highlight statistically significant
differences in item severity between subgroups. Examination of these
results indicates that the items were not significantly different in
severity between the two subgroups. In addition, there were both
positive and negative z statistics, indicating that although the
differences in item severity were not significant, there were some items
that were easier to endorse for Subgroup 1 and others that were easier
to endorse for Subgroup 2.

Figure 1: Plot of Standardized Differences for Items between Subgroups

To further explore the differences in item severity between the two
subgroups, Figure 2 shows a scatterplot of the item locations between
the two subgroups. In the plot, the item severity for Subgroup 1 is
shown on the x-axis, and the item severity for Subgroup 2 is shown on
the y-axis. Individual items are indicated using open circle plotting
symbols. A solid identity line is included to highlight deviations from
invariant item severities between the two groups: Points that fall below
this line indicate that items were easier to endorse (lower item
measures) for Subgroup 2, and points that fall above the line indicate
that items were easier to endorse (lower item measures) for Subgroup 1.
Dashed lines are also included to indicate a 95\% confidence interval
for the difference between the item measures, following Luppescu (1995).

Figure 2: Scatterplot of Subgroup-Specific Item Severities

Finally, Figure 3 is a bar plot that illustrates the direction and
magnitude of the differences in item severities between subgroups. In
the plot, each bar represents the difference in severity between
subgroups for an individual item, ordered by the item sequence in the
survey. Bars that point to the left of the plot indicate that the item
was easier to endorse for Subgroup 1, and bars that point to the right
of the plot indicate that the item was easier to endorse for Subgroup 2.
Dashed vertical lines are plotted that show values of +0.5 and -0.5
logits as an indicator of substantial differences in item severity
between subgroups.

Figure 3: Bar Plot of Differences In Item Severity Between Subgroups

\bookmarksetup{startatroot}

\hypertarget{discussion}{%
\chapter*{Discussion}\label{discussion}}
\addcontentsline{toc}{chapter}{Discussion}

The purpose of this study was to further validate and enhance the
Spiritual Harm and Abuse Scale (SHAS) with the use of the Rasch Rating
Scale model. The present study improves upon the sample dependency of
factor analysis results and contributes new understanding about the
psychometric characteristics of the SHAS.

In this study, we examined unidimensionality, Rasch person/item
separation reliability, model-data fit, distribution of the variable map
based on the locations of persons and items, and individual item
qualities of the SHAS. In terms of dimensionality, we found satisfactory
evidence to unidimensionality to apply the Rasch Rating Scale model. We
found the SHAS highly reliable using Rasch person/item separation
reliability statistics, indicating that most of the items are sensitive
enough to capture different levels of spiritual abuse and most persons'
observed values overlap with the expected values. Similarly, there was
close correspondence between the data and the Rasch Rating Scale model
except for the outliers in the XXX ends of the continuum. This indicates
reasonable model-data fit. Regarding the variable map, our results
indicate that the SHAS, overall, captures a range of item severities
that reflect the order structure of spiritual abuse. However, we also
found a shortage of items at the very high and low ends of the
continuum. The SHAS can capture people who have experienced moderate
severity of spiritual abuse, but it is not sensitive enough to
discriminate among people who have experienced very severe and less
severe spiritual abuse. \textbackslash{[}For example, based on our Rasch
Rating scale model analysis, someone with a measure of − 3.00 logits and
another with a measure of − 2.00 logits would be more likely to respond
to the SHAS in the same way because there is no item on the scale that
targets very low levels of spiritual abuse. Hence, we would not be able
to detect which person has relatively more or less spiritual abuse in
comparison to each other. It would be similarly impossible to
distinguish levels of spiritual abuse between a person with a measure of
3.00 logits and another with a measure of 2.00 logits, and decide who
has experienced more severe spiritual abuse due to lack of items at the
high end of the continuum.\textbackslash{]} Finally, in terms of item
quality, XX of the XX items exhibited good psychometric quality, whereas
XX items were misfits. XXX items were redundant because they provided
the exact same information with identical pairs of each item, displaying
the same contributions to the sensitivity of the SHAS. Hence, they can
be removed from the scale to shorten the duration of administration.

Use of the Rasch measurement model in our study provided a powerful
analysis of the functioning of the SHAS at the item level and offered
in-depth insights on applicability of the scale in wider contexts due to
its sample independent nature. Whereas factor analytic approaches such
as exploratory factor analysis and confirmatory factor analysis do not
provide invariant measurement due to their sample dependency, the Rasch
measurement model is independent from the items that form the scale and
independent from the sample that the scale is administered to. This
feature will increase the applicability of the scale in wider contexts
and replicability of the present study in different populations due to
the possibility of direct application of the standards developed in one
study to data administered in another study (Yan and Mok 2012).

In conclusion, the results of this study provide strong evidence for the
validation of the SHAS despite the need for considering the deletion of
XX items that are misfits and of XX items that are redundant. The
revision or development of new items can also be considered to capture
gradations of challenges at the very high and low ends of the continuum
to detect people with very high and low levels of spiritual abuse.

\hypertarget{implications-and-future-directions}{%
\section*{Implications and future
directions}\label{implications-and-future-directions}}
\addcontentsline{toc}{section}{Implications and future directions}

This study presents the validation of the SHAS using the Rasch
measurement model, for assessing spiritual abuse in adults. The findings
have important implications for clinicians and researchers who are
concerned about spiritual abuse. The finding that the majority of the
SHAS items were of high quality implies that the items provide adequate
information to measure moderate levels of spiritual abuse. However, we
found the SHAS less able to distinguish people with very high and low
levels of spiritual abuse. Hence, for now clinicians and researchers
should be cautious when interpreting their results for people who have
very high and low spiritual abuse levels.

We echo the same limitations of these data noted in @KochEdstrom2022.
The SHAS was developed on a largely white and Protestant/former
Protestant population. Possibly major types or sub-themes of spiritual
abuse not captured here are more prominent in other religious
traditions. This would be a fruitful area of future study. Research into
spiritual abuse among the LGBTQ+ community might also yield insights
specific to that population. Future research could also distinguish
normed scores of this scale for different populations and denominations.

The final screener has the potential to both improve therapy for
innumerable clients who have experienced negative religious experiences
and to contribute a knowledge base for any future clinical training
around working with spiritually abused clients. The scale may also
contribute to future research that could be critical in understanding
not only the mechanisms and experience of spiritual abuse but also
perhaps the enduring power of religion, both to heal and to harm.

\bookmarksetup{startatroot}

\hypertarget{references}{%
\chapter*{References}\label{references}}
\addcontentsline{toc}{chapter}{References}

\hypertarget{refs}{}
\begin{CSLReferences}{1}{0}
\leavevmode\vadjust pre{\hypertarget{ref-Andrich1978}{}}%
Andrich, David. 1978. {``A Rating Formulation for Ordered Response
Categories.''} \emph{Psychometrika} 43: 561--73.

\leavevmode\vadjust pre{\hypertarget{ref-BondFox2015}{}}%
Bond, Trevor G., and Christine M. Fox. 2015. \emph{Applying the Rasch
Model: Fundamental Measurement in the Human Sciences}. New York, New
York: Routledge.

\leavevmode\vadjust pre{\hypertarget{ref-R-mirt}{}}%
Chalmers, Phil. 2022. \emph{Mirt: Multidimensional Item Response
Theory}. \url{https://CRAN.R-project.org/package=mirt}.

\leavevmode\vadjust pre{\hypertarget{ref-R-lordif}{}}%
Choi, Seung W., with contributions from Laura E. Gibbons, and Paul K.
Crane. 2016. \emph{Lordif: Logistic Ordinal Regression Differential Item
Functioning Using IRT}. \url{https://CRAN.R-project.org/package=lordif}.

\leavevmode\vadjust pre{\hypertarget{ref-EmbretsonReise2000}{}}%
Embretson, Susan E., and Steven P. Reise. 2000. \emph{Item Response
Theory for Psychologists}. Mahwah, New Jerey: Lawrence Erlbaum
Associates.

\leavevmode\vadjust pre{\hypertarget{ref-Keller2016}{}}%
Keller, K. H. 2016. {``Development of a Spiritual Abuse
Questionnaire.''} PhD thesis, Texas Woman's University.

\leavevmode\vadjust pre{\hypertarget{ref-KochEdstrom2022}{}}%
Koch, Dan, and Leihua Edstrom. 2022. {``Development of the Spiritual
Harm and Abuse Scale.''} \emph{Journal for the Scientific Study of
Religion}.

\leavevmode\vadjust pre{\hypertarget{ref-R-eRm}{}}%
Mair, Patrick, Reinhold Hatzinger, and Marco J. Maier. 2021. \emph{eRm:
Extended Rasch Modeling}. \url{https://CRAN.R-project.org/package=eRm}.

\leavevmode\vadjust pre{\hypertarget{ref-R-base}{}}%
R Core Team. 2022. \emph{R: A Language and Environment for Statistical
Computing}. Vienna, Austria: R Foundation for Statistical Computing.
\url{https://www.R-project.org/}.

\leavevmode\vadjust pre{\hypertarget{ref-Rasch1960}{}}%
Rasch, Georg. 1960. \emph{Probabilistic Models for Some Intelligence and
Attainment Tests}. Chicago, IL: University of Chicago Press.

\leavevmode\vadjust pre{\hypertarget{ref-Reckase1979}{}}%
Reckase, Mark D. 1979. {``Unifactor Latent Trait Models Applied to
Multifactor Tests: Results and Implications.''} \emph{Journal of
Educational Statistics} 4: 207--30.

\leavevmode\vadjust pre{\hypertarget{ref-ReiseBonifayHaviland2013}{}}%
Reise, Steven P., Wes E. Bonifay, and Mark G. Haviland. 2013. {``Scoring
and Modeling Psychological Measures in the Presence of
Multidimensionality.''} \emph{Journal of Personality Assessment} 95 (2):
129--40. \url{https://doi.org/10.1080/00223891.2012.725437}.

\leavevmode\vadjust pre{\hypertarget{ref-ReiseMooreHaviland2013}{}}%
Reise, Steven P., Tyler M. Moore, and Mark G. Haviland. 2013.
{``Applying Unidimensional Item Response Theory Models to Psychological
Data.''} In \emph{APA Handbook of Testing and Assessment in Psychology:
Vol. 1. Test Theory and Testing and Assessment in Industrial and
Organizational Psychology}, 101--19.

\leavevmode\vadjust pre{\hypertarget{ref-ReiseWaller2009}{}}%
Reise, Steven P., and Niels G. Waller. 2009. {``Item Response Theory and
Clinical Measurement.''} \emph{Annual Review of Clinical Psychology} 5:
27--48. \url{https://doi.org/10.1146/annurev.clinpsy.032408.153553}.

\leavevmode\vadjust pre{\hypertarget{ref-R-psych}{}}%
Revelle, William. 2022. \emph{Psych: Procedures for Psychological,
Psychometric, and Personality Research}.
\href{https://personality-project.org/r/psych/\%0Ahttps://personality-project.org/r/psych-manual.pdf}{https://personality-project.org/r/psych/
https://personality-project.org/r/psych-manual.pdf}.

\leavevmode\vadjust pre{\hypertarget{ref-R-TAM}{}}%
Robitzsch, Alexander, Thomas Kiefer, and Margaret Wu. 2022. \emph{TAM:
Test Analysis Modules}. \url{https://CRAN.R-project.org/package=TAM}.

\end{CSLReferences}



\end{document}
